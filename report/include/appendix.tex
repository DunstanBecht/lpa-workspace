\nocite{*}

\begin{appendix}

\section{Mathematical developments}

\subsection{Random positioning optimization}

This section presents the methods used for the random positioning of points that must respect certain properties.
A first approach would be to hide the points not respecting the required conditions.
It is more efficient to draw only points having already the required properties.

\subsubsection{Continuous uniform distribution in a circle}\label{sec:circle-random-position}

A polar coordinate system \( (\varphi, r) \) is used in a disk of radius \( R_0 \) in which random positions are to be drawn.
The probability density of presence of a point in the disk is \( 1 / \left( \pi {R_0}^2 \right) \).
At each random draw, the probability to find the dislocation in the surface element \( r dr d\varphi\) is then:

\begin{equation}
  \frac{r dr d\varphi}{\pi {R_0}^2} =
    \frac{2 r}{{R_0}^2} dr \frac{1}{2 \pi} d\varphi
\end{equation}

\bigskip

Let \( \phi \) and \( R \) be two random variables such that \( \phi(\omega) \in [0, 2\pi[ \) and \( R(\omega) \in [0, R_0] \). Let \( f_\phi \) and \( f_R \) be the probability density functions of \( \phi \) and \( R \), respectively.

\begin{equation}
  f_\phi(\varphi) =
    \frac{1}{2 \pi}
\end{equation}

\begin{equation}
  f_R(r) =
    \frac{2r}{{R_0}^2}
\end{equation}

\medskip

Let \( F_\phi \) and  \( F_R \) be the  cumulative distribution functions of \( \phi \) and \( R \), respectively.

\begin{equation}
  F_\phi(\varphi) =
    \mathbb{P}(\phi \leq \varphi) =
      \int_0^\varphi f_\phi(t) dt = \frac{\varphi }{2 \pi}
\end{equation}

\begin{equation}
  F_R(r) =
    \mathbb{P}(R \leq r) =
      \int_0^r f_R(t) dt = \left( \frac{r}{R_0} \right)^2
\end{equation}

\medskip

Let \( X_\phi \) and \( X_R \) be two new random variables defined using \( \phi \) and \( R \) such that \( X_\phi \) and \( X_R \) follow a uniform law \( U(0, 1) \).

\begin{equation}
  X_\phi =
    \frac{\phi}{2 \pi}
\end{equation}

\begin{equation}
  X_R =
    \left( \frac{R}{R_0} \right)^2
\end{equation}

\medskip

Since \( X_\phi \sim U(0, 1) \) and \( X_R \sim U(0, 1) \), the random position \( (\varphi, r)\) of a point on the disk can be choosen with the sampling \( ( x_\phi, x_R ) \) of two uniformly distributed random variables \( X_\phi \) and \( X_R \):

\begin{equation}\label{eq:circle-random-position-theta}
  \varphi =
    2 \pi x_\phi
\end{equation}

\begin{equation}\label{eq:circle-random-position-r}
  r =
    R_0 \sqrt{x_R}
\end{equation}

\subsubsection{Drawing of points in cell walls}\label{sec:cell-random-position}

\bigfig{fig:cell-random-position}{insert/positions}{walls}{Division of the surface occupied by the walls}

To randomly select points in the cell borders, 4 virtual rectangles \( A, B, C, D \) are created in each cell. If the probability of presence in the cell is given by \( P = \gls{area}_{\text{cell}} / \gls{area}_{\gls{roi}} \), then the probability of presence in the rectangles is given by \( P_A = P_B = P_C = P_D = P / 4 \).
The position in the selected rectangle is then chosen by applying a uniform continuous law.

\subsection{Overlapping}\label{sec:overlapping}

\subsubsection{Circular segment area}

For the area shaded in \figref{fig:overlapping:circular-segment} the formula is obtained by calculating the angle between d and r.

\begin{minipage}{0.5\linewidth}
  \bigfig{fig:overlapping:circular-segment}{insert/overlapping}{circular_segment}{Circular segment area}
\end{minipage}%
\begin{minipage}{0.5\linewidth}
  \begin{equation}\label{eq:overlapping:circular-segment}
    \cirseg(r, d) =
      r^2 \arccos\left( \frac{d}{r} \right) - d \sqrt{r^2 - d^2}
  \end{equation}
\end{minipage}%

\subsubsection{Intersection area of two circles}\label{sec:circle-circle-intersection}

The area of intersection of two circles of radius \( r \) and \( R \) whose centers are spaced by a distance \( d \) as shown in \figref{fig:circle-circle-intersection} can be calculated using equation \eqref{eq:overlapping:circular-segment}. The expression is given by the function \( \circir (r, R, d) \) in equation \eqref{eq:circle-circle-intersection}.

\begin{align}
  \gls{area}_0 (r, R, d) & =
    \textstyle
    \cirseg \left(r, \frac{d^2 + r^2 - R^2}{2 d} \right) + \cirseg \left( R, \frac{d^2 + R^2 - r^2}{2 d} \right)
  \\[1mm]
  \gls{area}_0 (r, R, d) & =
    \textstyle
    r^2 \arccos \left( \frac{d^2 + r^2 - R^2}{2 d r} \right)
    + R^2 \arccos \left( \frac{d^2 + R^2 - r^2}{2 d R} \right)
    - \frac{\sqrt{(-d+r+R)(d+r-R)(d-r+R)(d+r+R)}}{2}
\end{align}

\begin{minipage}{0.5\linewidth}
  \bigfig{fig:circle-circle-intersection}{insert/overlapping}{circle_circle}{Intersection of two circles}
\end{minipage}%
\begin{minipage}{0.5\linewidth}
  \begin{equation}\label{eq:circle-circle-intersection}
    \circir (r, R, d) =
    \begin{cases}
      0 & \text{for } r + R < d \\[2mm]
      \pi r^2 & \text{for } d + r < R \\[2mm]
      \pi R^2 & \text{for } d + R < r \\[2mm]
      \gls{area}_0 (r, R, d) & \text{else}
    \end{cases}
  \end{equation}
\end{minipage}%

\subsubsection{Intersection area of a circle and a square}\label{sec:circle-square-intersection}

\bigfig{fig:overlapping:circle-square-examples}{insert/overlapping}{circle_square}{Some possibilities of overlapping of a square and a circle}

In order to calculate the area \cirsqr \ of intersection between a circle of radius \( r \) and a square of side \( S \) one would have to distinguish many cases of the circle overtaking the right, left and bottom of the square.
To simplify the distinction of cases the study is limited to one quadrant of the circle in a first time.
The area of the quadrant outside the square is calculated and the contribution of each quadrant is summed and subtracted from the area of the circle to obtain the total area of intersection.
\figref{fig:overlapping:circle-square-quadrants} show the different considered possibilities in a circle quadrant.

\bigfig{fig:overlapping:circle-square-quadrants}{insert/overlapping}{quadrants}{Possibilities illustrated in the upper right circle quadrant}

\bigskip

The occurrence of the cases presented in \figref{fig:overlapping:circle-square-quadrants} is formally defined as a function of \( d_1 \), \( d_2 \) and \( r \) in equations \eqref{eq:quadrant-case-1}, \eqref{eq:quadrant-case-2} and \eqref{eq:quadrant-case-3}.

\begin{align}
  C1 &\Longleftrightarrow \left( \neg C3 \right) \wedge \left( d_1 < r \right) \label{eq:quadrant-case-1} \\
  C2 &\Longleftrightarrow \left( \neg C3 \right) \wedge \left( d_2 < r \right) \label{eq:quadrant-case-2} \\
  C3 &\Longleftrightarrow \left( d_1^2 + d_2^2 < r^2 \right) \label{eq:quadrant-case-3}
\end{align}

\medskip

The areas grayed in \figref{fig:overlapping:circle-square-quadrants} are expressed in \eqref{eq:overlapping:area-12} and \eqref{eq:overlapping:area-3} using \( \cirseg(r, d) \) defined in equation \eqref{eq:overlapping:circular-segment}.
\begin{align}
\gls{area}_{1 \lor 2} (r, d) &= \frac{\cirseg(r, d)}{2} \label{eq:overlapping:area-12} \ \text{for } C1 \text{ or } C2 \\
\gls{area}_3 (r, d_1, d_2) &= \frac{\pi r^2}{4} - d_1 d_2 \label{eq:overlapping:area-3} \ \text{for } C3
\end{align}

\medskip

The area \( \gls{area}_Q (r, d_1, d_2) \) outside a quadrant is then given in equation \eqref{eq:outside-quadrant}.

\begin{equation}\label{eq:outside-quadrant}
  \gls{area}_Q (r, d_1, d_2) =
    \gls{indfun}_{C1} \gls{area}_{1 \lor 2} (r, d_1) +
    \gls{indfun}_{C2} \gls{area}_{1 \lor 2} (r, d_2) +
    \gls{indfun}_{C3} \gls{area}_{3} (r, d_1, d_2)
\end{equation}

\medskip

Finally, to obtain the area \( \cirsqr (r, S, x, y) \) of intersection between a circle of radius \( r \) centered in \( (x, y) \) and a square of side \( S \), the contribution of each quadrant is taken into account in equation \eqref{eq:circle-square-intersection}.

\begin{equation}\label{eq:circle-square-intersection}
  \cirsqr (r, S, x, y) =
    \pi r^2
    - \gls{area}_Q (r,     x,     y)
    - \gls{area}_Q (r, S - x,     y)
    - \gls{area}_Q (r,     x, S - y)
    - \gls{area}_Q (r, S - x, S - y)
\end{equation}

\subsubsection{Average intersection area of two circles}

The expected value of intersection of two circles of radius \( r \) and \( R \) whose centers are spaced by a random distance \( D \in [0, R] \) is given by the function \( \gls{expval} \left( \circir (r, R, D) \right) \) defined in equation \eqref{eq:mean-circle-circle-intersection}.
For a uniform distribution of points, \( \textstyle D = R \sqrt{X} \) with \( X \) following a uniform law \( U(0, 1) \) as explained in \ref{sec:circle-random-position}.

\begin{equation}\label{eq:mean-circle-circle-intersection}
  \gls{expval} \left( \circir(r, R, D) \right) =
    \int_{x=0}^1 \circir (r, R, R \sqrt{x}) dx
\end{equation}

\bigfig{fig:mean-circle-circle-intersection}{insert/overlapping}{mean_circle_circle}{Comparison of the analytical formulation \eqref{eq:mean-circle-circle-intersection} of \( \gls{expval} \left( \circir \right) \) with a simulation}

\subsubsection{Average intersection area of a circle and a square}

Let \( S \) be the side of a square and \( r \) the circle of a radius.
Let \( X \) and \( Y \) be two random variables following a uniform law \( U(0, S) \).
Here it is sought the analytical expression of \( \gls{expval} \left( \cirsqr (r, S, X, Y) \right) \).
\figref{fig:overlapping:application-areas} show the areas of occurrence of the cases presented in \figref{fig:overlapping:circle-square-quadrants}.
When the center \( ( X, Y ) \) of the circle is choosen in a gray area of \figref{fig:overlapping:application-areas}, then we calculate the corresponding air in \figref{fig:overlapping:circle-square-quadrants}.
The probability that the center of the circle is chosen in a gray area is proportional to the area of the area since the law is uniform.
But the air outside the square depends on the position.
It will therefore be necessary to integrate these areas over the positions of occurrence.

\bigfig{fig:overlapping:application-areas}{insert/overlapping}{application_areas}{Case application areas for various radius values}

\medskip

It is defined limits of integration from \figref{fig:overlapping:application-areas}.

\begin{align}
  \varphi_1 &=
    \arccos \left( \frac{S}{\max(r, S)} \right) \\
  \varphi_2 &=
    \arctan \left( \frac{S}{\min(r, S)} \right) \\
  \varphi_3 &=
    \arcsin \left( \frac{S}{\max(r, S)} \right) \\
  x_1 (\varphi) &=
    \frac{\min(r, S)}{\cos{\varphi}} \\
  x_2 (\varphi) &=
    \frac{S}{\sin{\varphi}}
\end{align}

\medskip

The contributions for one quadrant of the different cases can then be expressed.

\begin{align}
  \gls{expval} \left( \gls{area}_{1 \lor 2} \right) &=
    \frac{1}{S^2} \left(
    \int_{\varphi = \varphi_1}^{\varphi_2}
    \int_{x=r}^{x_1(\varphi)}
    \gls{area}_{1 \lor 2} (r, x \cos \left( \varphi) \right)
    x dx d\varphi
    +
    \int_{\varphi = \varphi_2}^{\varphi_3}
    \int_{x=r}^{x_2(\varphi)}
    \gls{area}_{1 \lor 2} (r, x \cos \left( \varphi) \right)
    x dx d\varphi
    \right)
  \\
  \gls{expval} \left( \gls{area}_{3} \right) &=
    \frac{1}{S^2} \left(
    \int_{\varphi = \varphi_1}^{\varphi_3}
    \int_{x=0}^{r}
    \gls{area}_{3} \left( r, x \sin(\varphi), x \cos(\varphi) \right)
    x dx d\varphi
    +
    2 \int_{y = 0}^{S}
    \int_{x=0}^{\tan(\varphi_1)y}
    \gls{area}_{3} \left( r, x, y \right)
    dx dy
    \right)
\end{align}

\medskip

Finally, for the mean intersection area, the contribution of each quadrant is taken into account.

\begin{equation}\label{eq:mean-circle-square-intersection}
  \gls{expval} \left( \cirsqr (r, S, X, Y) \right) =
    \begin{cases}
      0 &\text{if } r = 0
      \\
      \pi r^2 - 4 \left(2 \gls{expval} \left( \gls{area}_{1 \lor 2} \right) +\gls{expval} \left( \gls{area}_{3} \right)\right) &\text{if } r \in \ ] 0, \sqrt{2} S [
      \\
      S^2 &\text{if } r \geq \sqrt{2} S
    \end{cases}
\end{equation}

\bigfig{fig:mean-circle-square-intersection}{insert/overlapping}{mean_circle_square}{Comparison of the analytical formulation \eqref{eq:mean-circle-square-intersection} of \( \gls{expval} \left( \cirsqr \right) \) with a simulation}

\section{Data}\label{sec:data}

The construction and analysis of the data produced for the study was automated.
Everything was done from a same directory called \textit{workspace}.
This repository contains the parameters and the the scripts that will allow you to replicate the results obtained.
The installation of the three packages mentioned above is necessary for the execution of the scripts.

\bigskip

You can find the scripts and documentations here: \github{lpa-workspace}

\bigskip

You can also clone the project in tour local directory with git:

\begin{tcolorbox}[width=\linewidth, title=shell]
\begin{verbatim}
$ cd <the-place-you-want-to-clone-the-repository>
$ git clone https://github.com/DunstanBecht/lpa-workspace.git
\end{verbatim}
\end{tcolorbox}

\bigskip

The following pages provide a summary of the data generated for each distribution studied.

\newpage

\begin{multicols}{2}

\subsection{RDD}
\subsubsection{RDD-5e13}\label{RDD-5e13}
\res{13}{rho5e13m-2_square_3200nm_RDD_d5e-5nm-2_S0.pdf}{100_rho5e13m-2_square_3200nm_RDD_d5e-5nm-2_edge_S0_PBC1_output_analysis}{200000_rho5e13m-2_RDD_d5e-5nm-2_square_3200nm_S0_KKKK_GBB.pdf}{200000_rho5e13m-2_RDD_d5e-5nm-2_square_3200nm_S0_gggg_GBB.pdf}{200000_rho5e13m-2_RDD_d5e-5nm-2_square_3200nm_S0_GaGs_GBB.pdf}

\subsubsection{RDD-5e14}\label{RDD-5e14}
\res{14}{rho5e14m-2_square_3200nm_RDD_d5e-4nm-2_S0.pdf}{100_rho5e14m-2_square_3200nm_RDD_d5e-4nm-2_edge_S0_PBC1_output_analysis}{20000_rho5e14m-2_RDD_d5e-4nm-2_square_3200nm_S0_KKKK_GBB.pdf}{20000_rho5e14m-2_RDD_d5e-4nm-2_square_3200nm_S0_gggg_GBB.pdf}{20000_rho5e14m-2_RDD_d5e-4nm-2_square_3200nm_S0_GaGs_GBB.pdf}

\subsection{RRDD-E}
\subsubsection{RRDD-E-5e13-A}\label{RRDD-E-5e13-A}
\res{13}{rho5e13m-2_square_3200nm_RRDD-E_d5e-5nm-2_s0200nm_S0.pdf}{100_rho5e13m-2_square_3200nm_RRDD-E_d5e-5nm-2_s0200nm_edge_S0_PBC1_output_analysis}{200000_rho5e13m-2_RRDD-E_d5e-5nm-2_s0200nm_square_3200nm_S0_KKKK_GBB.pdf}{200000_rho5e13m-2_RRDD-E_d5e-5nm-2_s0200nm_square_3200nm_S0_gggg_GBB.pdf}{200000_rho5e13m-2_RRDD-E_d5e-5nm-2_s0200nm_square_3200nm_S0_GaGs_GBB.pdf}

\subsubsection{RRDD-E-5e14-A}\label{RRDD-E-5e14-A}
\res{14}{rho5e14m-2_square_3200nm_RRDD-E_d5e-4nm-2_s0200nm_S0.pdf}{100_rho5e14m-2_square_3200nm_RRDD-E_d5e-4nm-2_s0200nm_edge_S0_PBC1_output_analysis}{20000_rho5e14m-2_RRDD-E_d5e-4nm-2_s0200nm_square_3200nm_S0_KKKK_GBB.pdf}{20000_rho5e14m-2_RRDD-E_d5e-4nm-2_s0200nm_square_3200nm_S0_gggg_GBB.pdf}{20000_rho5e14m-2_RRDD-E_d5e-4nm-2_s0200nm_square_3200nm_S0_GaGs_GBB.pdf}

\subsubsection{RRDD-E-5e13-B}\label{RRDD-E-5e13-B}
\res{13}{rho5e13m-2_square_3200nm_RRDD-E_d5e-5nm-2_s0400nm_S0.pdf}{100_rho5e13m-2_square_3200nm_RRDD-E_d5e-5nm-2_s0400nm_edge_S0_PBC1_output_analysis}{200000_rho5e13m-2_RRDD-E_d5e-5nm-2_s0400nm_square_3200nm_S0_KKKK_GBB.pdf}{200000_rho5e13m-2_RRDD-E_d5e-5nm-2_s0400nm_square_3200nm_S0_gggg_GBB.pdf}{200000_rho5e13m-2_RRDD-E_d5e-5nm-2_s0400nm_square_3200nm_S0_GaGs_GBB.pdf}

\subsubsection{RRDD-E-5e14-B}\label{RRDD-E-5e14-B}
\res{14}{rho5e14m-2_square_3200nm_RRDD-E_d5e-4nm-2_s0400nm_S0.pdf}{100_rho5e14m-2_square_3200nm_RRDD-E_d5e-4nm-2_s0400nm_edge_S0_PBC1_output_analysis}{20000_rho5e14m-2_RRDD-E_d5e-4nm-2_s0400nm_square_3200nm_S0_KKKK_GBB.pdf}{20000_rho5e14m-2_RRDD-E_d5e-4nm-2_s0400nm_square_3200nm_S0_gggg_GBB.pdf}{20000_rho5e14m-2_RRDD-E_d5e-4nm-2_s0400nm_square_3200nm_S0_GaGs_GBB.pdf}

\subsection{RRDD-R}
\subsubsection{RRDD-R-5e13-A}\label{RRDD-R-5e13-A}
\res{13}{rho5e13m-2_square_3200nm_RRDD-R_d5e-5nm-2_s0200nm_S0.pdf}{100_rho5e13m-2_square_3200nm_RRDD-R_d5e-5nm-2_s0200nm_edge_S0_PBC1_output_analysis}{200000_rho5e13m-2_RRDD-R_d5e-5nm-2_s0200nm_square_3200nm_S0_KKKK_GBB.pdf}{200000_rho5e13m-2_RRDD-R_d5e-5nm-2_s0200nm_square_3200nm_S0_gggg_GBB.pdf}{200000_rho5e13m-2_RRDD-R_d5e-5nm-2_s0200nm_square_3200nm_S0_GaGs_GBB.pdf}

\subsubsection{RRDD-R-5e14-A}\label{RRDD-R-5e14-A}
\res{14}{rho5e14m-2_square_3200nm_RRDD-R_d5e-4nm-2_s0200nm_S0.pdf}{100_rho5e14m-2_square_3200nm_RRDD-R_d5e-4nm-2_s0200nm_edge_S0_PBC1_output_analysis}{20000_rho5e14m-2_RRDD-R_d5e-4nm-2_s0200nm_square_3200nm_S0_KKKK_GBB.pdf}{20000_rho5e14m-2_RRDD-R_d5e-4nm-2_s0200nm_square_3200nm_S0_gggg_GBB.pdf}{20000_rho5e14m-2_RRDD-R_d5e-4nm-2_s0200nm_square_3200nm_S0_GaGs_GBB.pdf}

\subsubsection{RRDD-R-5e13-B}\label{RRDD-R-5e13-B}
\res{13}{rho5e13m-2_square_3200nm_RRDD-R_d5e-5nm-2_s0400nm_S0.pdf}{100_rho5e13m-2_square_3200nm_RRDD-R_d5e-5nm-2_s0400nm_edge_S0_PBC1_output_analysis}{200000_rho5e13m-2_RRDD-R_d5e-5nm-2_s0400nm_square_3200nm_S0_KKKK_GBB.pdf}{200000_rho5e13m-2_RRDD-R_d5e-5nm-2_s0400nm_square_3200nm_S0_gggg_GBB.pdf}{200000_rho5e13m-2_RRDD-R_d5e-5nm-2_s0400nm_square_3200nm_S0_GaGs_GBB.pdf}

\subsubsection{RRDD-R-5e14-B}\label{RRDD-R-5e14-B}
\res{14}{rho5e14m-2_square_3200nm_RRDD-R_d5e-4nm-2_s0400nm_S0.pdf}{100_rho5e14m-2_square_3200nm_RRDD-R_d5e-4nm-2_s0400nm_edge_S0_PBC1_output_analysis}{20000_rho5e14m-2_RRDD-R_d5e-4nm-2_s0400nm_square_3200nm_S0_KKKK_GBB.pdf}{20000_rho5e14m-2_RRDD-R_d5e-4nm-2_s0400nm_square_3200nm_S0_gggg_GBB.pdf}{20000_rho5e14m-2_RRDD-R_d5e-4nm-2_s0400nm_square_3200nm_S0_GaGs_GBB.pdf}

\subsection{RCDD-E}
\subsubsection{RCDD-E-5e13}\label{RCDD-E-5e13}
\res{13}{rho5e13m-2_square_3200nm_RCDD-E_d5e-5nm-2_s0800nm_t084nm_S0.pdf}{100_rho5e13m-2_square_3200nm_RCDD-E_d5e-5nm-2_s0800nm_t084nm_edge_S0_PBC1_output_analysis}{200000_rho5e13m-2_RCDD-E_d5e-5nm-2_s0800nm_t084nm_square_3200nm_S0_KKKK_GBB.pdf}{200000_rho5e13m-2_RCDD-E_d5e-5nm-2_s0800nm_t084nm_square_3200nm_S0_gggg_GBB.pdf}{200000_rho5e13m-2_RCDD-E_d5e-5nm-2_s0800nm_t084nm_square_3200nm_S0_GaGs_GBB.pdf}

\subsubsection{RCDD-E-5e14}\label{RCDD-E-5e14}
\res{14}{rho5e14m-2_square_3200nm_RCDD-E_d5e-4nm-2_s0800nm_t084nm_S0.pdf}{100_rho5e14m-2_square_3200nm_RCDD-E_d5e-4nm-2_s0800nm_t084nm_edge_S0_PBC1_output_analysis}{20000_rho5e14m-2_RCDD-E_d5e-4nm-2_s0800nm_t084nm_square_3200nm_S0_KKKK_GBB.pdf}{20000_rho5e14m-2_RCDD-E_d5e-4nm-2_s0800nm_t084nm_square_3200nm_S0_gggg_GBB.pdf}{20000_rho5e14m-2_RCDD-E_d5e-4nm-2_s0800nm_t084nm_square_3200nm_S0_GaGs_GBB.pdf}

\subsection{RCDD-R}
\subsubsection{RCDD-R-5e13}\label{RCDD-R-5e13}
\res{13}{rho5e13m-2_square_3200nm_RCDD-R_d5e-5nm-2_s0800nm_t084nm_S0.pdf}{100_rho5e13m-2_square_3200nm_RCDD-R_d5e-5nm-2_s0800nm_t084nm_edge_S0_PBC1_output_analysis}{200000_rho5e13m-2_RCDD-R_d5e-5nm-2_s0800nm_t084nm_square_3200nm_S0_KKKK_GBB.pdf}{200000_rho5e13m-2_RCDD-R_d5e-5nm-2_s0800nm_t084nm_square_3200nm_S0_gggg_GBB.pdf}{200000_rho5e13m-2_RCDD-R_d5e-5nm-2_s0800nm_t084nm_square_3200nm_S0_GaGs_GBB.pdf}

\subsubsection{RCDD-R-5e14}\label{RCDD-R-5e14}
\res{14}{rho5e14m-2_square_3200nm_RCDD-R_d5e-4nm-2_s0800nm_t084nm_S0.pdf}{100_rho5e14m-2_square_3200nm_RCDD-R_d5e-4nm-2_s0800nm_t084nm_edge_S0_PBC1_output_analysis}{20000_rho5e14m-2_RCDD-R_d5e-4nm-2_s0800nm_t084nm_square_3200nm_S0_KKKK_GBB.pdf}{20000_rho5e14m-2_RCDD-R_d5e-4nm-2_s0800nm_t084nm_square_3200nm_S0_gggg_GBB.pdf}{20000_rho5e14m-2_RCDD-R_d5e-4nm-2_s0800nm_t084nm_square_3200nm_S0_GaGs_GBB.pdf}

\subsection{RCDD-D}
\subsubsection{RCDD-D-5e13-A}\label{RCDD-D-5e13-A}
\res{13}{rho5e13m-2_square_3200nm_RCDD-D_d5e-5nm-2_s0800nm_t084nm_l071nm_S0.pdf}{100_rho5e13m-2_square_3200nm_RCDD-D_d5e-5nm-2_s0800nm_t084nm_l071nm_edge_S0_PBC1_output_analysis}{200000_rho5e13m-2_RCDD-D_d5e-5nm-2_s0800nm_t084nm_l071nm_square_3200nm_S0_KKKK_GBB.pdf}{200000_rho5e13m-2_RCDD-D_d5e-5nm-2_s0800nm_t084nm_l071nm_square_3200nm_S0_gggg_GBB.pdf}{200000_rho5e13m-2_RCDD-D_d5e-5nm-2_s0800nm_t084nm_l071nm_square_3200nm_S0_GaGs_GBB.pdf}

\subsubsection{RCDD-D-5e14-A}\label{RCDD-D-5e14-A}
\res{14}{rho5e14m-2_square_3200nm_RCDD-D_d5e-4nm-2_s0800nm_t084nm_l022nm_S0.pdf}{100_rho5e14m-2_square_3200nm_RCDD-D_d5e-4nm-2_s0800nm_t084nm_l022nm_edge_S0_PBC1_output_analysis}{20000_rho5e14m-2_RCDD-D_d5e-4nm-2_s0800nm_t084nm_l022nm_square_3200nm_S0_KKKK_GBB.pdf}{20000_rho5e14m-2_RCDD-D_d5e-4nm-2_s0800nm_t084nm_l022nm_square_3200nm_S0_gggg_GBB.pdf}{20000_rho5e14m-2_RCDD-D_d5e-4nm-2_s0800nm_t084nm_l022nm_square_3200nm_S0_GaGs_GBB.pdf}

\subsubsection{RCDD-D-5e13-B}\label{RCDD-D-5e13-B}
\res{13}{rho5e13m-2_square_3200nm_RCDD-D_d5e-5nm-2_s0800nm_t084nm_l141nm_S0.pdf}{100_rho5e13m-2_square_3200nm_RCDD-D_d5e-5nm-2_s0800nm_t084nm_l141nm_edge_S0_PBC1_output_analysis}{200000_rho5e13m-2_RCDD-D_d5e-5nm-2_s0800nm_t084nm_l141nm_square_3200nm_S0_KKKK_GBB.pdf}{200000_rho5e13m-2_RCDD-D_d5e-5nm-2_s0800nm_t084nm_l141nm_square_3200nm_S0_gggg_GBB.pdf}{200000_rho5e13m-2_RCDD-D_d5e-5nm-2_s0800nm_t084nm_l141nm_square_3200nm_S0_GaGs_GBB.pdf}

\subsubsection{RCDD-D-5e14-B}\label{RCDD-D-5e14-B}
\res{14}{rho5e14m-2_square_3200nm_RCDD-D_d5e-4nm-2_s0800nm_t084nm_l045nm_S0.pdf}{100_rho5e14m-2_square_3200nm_RCDD-D_d5e-4nm-2_s0800nm_t084nm_l045nm_edge_S0_PBC1_output_analysis}{20000_rho5e14m-2_RCDD-D_d5e-4nm-2_s0800nm_t084nm_l045nm_square_3200nm_S0_KKKK_GBB.pdf}{20000_rho5e14m-2_RCDD-D_d5e-4nm-2_s0800nm_t084nm_l045nm_square_3200nm_S0_gggg_GBB.pdf}{20000_rho5e14m-2_RCDD-D_d5e-4nm-2_s0800nm_t084nm_l045nm_square_3200nm_S0_GaGs_GBB.pdf}

\subsubsection{RCDD-D-5e13-C}\label{RCDD-D-5e13-C}
\res{13}{rho5e13m-2_square_3200nm_RCDD-D_d5e-5nm-2_s0800nm_t084nm_l283nm_S0.pdf}{100_rho5e13m-2_square_3200nm_RCDD-D_d5e-5nm-2_s0800nm_t084nm_l283nm_edge_S0_PBC1_output_analysis}{200000_rho5e13m-2_RCDD-D_d5e-5nm-2_s0800nm_t084nm_l283nm_square_3200nm_S0_KKKK_GBB.pdf}{200000_rho5e13m-2_RCDD-D_d5e-5nm-2_s0800nm_t084nm_l283nm_square_3200nm_S0_gggg_GBB.pdf}{200000_rho5e13m-2_RCDD-D_d5e-5nm-2_s0800nm_t084nm_l283nm_square_3200nm_S0_GaGs_GBB.pdf}

\subsubsection{RCDD-D-5e14-C}\label{RCDD-D-5e14-C}
\res{14}{rho5e14m-2_square_3200nm_RCDD-D_d5e-4nm-2_s0800nm_t084nm_l089nm_S0.pdf}{100_rho5e14m-2_square_3200nm_RCDD-D_d5e-4nm-2_s0800nm_t084nm_l089nm_edge_S0_PBC1_output_analysis}{20000_rho5e14m-2_RCDD-D_d5e-4nm-2_s0800nm_t084nm_l089nm_square_3200nm_S0_KKKK_GBB.pdf}{20000_rho5e14m-2_RCDD-D_d5e-4nm-2_s0800nm_t084nm_l089nm_square_3200nm_S0_gggg_GBB.pdf}{20000_rho5e14m-2_RCDD-D_d5e-4nm-2_s0800nm_t084nm_l089nm_square_3200nm_S0_GaGs_GBB.pdf}



\null

\end{multicols}

\section{Glossary}

\printnoidxglossaries

\printbibliography[heading=bibintoc, heading=bibnumbered]

\end{appendix}
