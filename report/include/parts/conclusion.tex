\section{Conclusions}

Several conclusions can be drawn from the synthesis in section \ref{sec:synthesis}.

\begin{itemize}
\item The simplified asymptotic models of \gls{guw} and \gls{w} are equivalent to \gls{kr} for the density prediction.
\item The parameter \( s \sqrt{\gls{dst}} \) is very important for the determination of the error for both distribution \gls{rrdd} and \gls{rcdd}.
\item The parameter \( \gls{frrstp} \sqrt{\gls{dst}} \) is also important. It explains the way the models become incorrect at high densities when the value of \gls{frrstp} has a lower bound.
\item \gls{w} should not be applied beyond the \gls{rrdd} hypotheses.
\item \gls{kr} usualy gets better results than \gls{w} and \gls{guw}.
\end{itemize}

The generation and analysis of the data produced for the study was automated.
Everything was done from a same directory called \textit{workspace}.
This repository contains the parameters and the the scripts that will allow you to replicate the results obtained.

\bigskip

You can find the scripts and documentations here: \github{lpa-workspace}

\bigskip

You can also clone the project in tour local directory with git:

\begin{tcolorbox}[width=\linewidth, title=shell]
\begin{verbatim}
$ cd <the-place-you-want-to-clone-the-repository>
$ git clone https://github.com/DunstanBecht/lpa-workspace.git
\end{verbatim}
\end{tcolorbox}
