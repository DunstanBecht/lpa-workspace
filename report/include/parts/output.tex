\section{Output}

The simulation program gives the coefficients \gls{frrcfa} and \gls{frrcfb} of the Fourier transform of \gls{difint}.
It is then possible to calculate the latter for each harmonic \gls{hmc} and value of \gls{frrvar} by averaging the coefficients over all the simulation outputs from the same set of starting parameters.

\begin{equation}\label{eq:fourier-amplitude}
  \gls{frrtrf}(\gls{frrvar}) =
    \hat{\gls{expval}} \left( \gls{frrcfa}(\gls{frrvar}) \right) + i \times \hat{\gls{expval}} \left( \gls{frrcfb}(\gls{frrvar}) \right)
\end{equation}

\medskip

\gls{frrtrf} is complex.
It is either its module or its real part that will be used.
The functions can be represented according to two methods as shown in \figref{fig:rdd-output}.
In \gls{r1} the functions are represented in logarithmic scale as a function of \gls{frrvar}.
In \gls{r2} the functions are represented through the transformation \( \ln\left( \cdot \right) / \gls{frrvar}^2 \) as a function of \gls{frrvar} in logarithmic scale.
With \gls{r2} a linear behavior can be observed for the asymptotic values of \gls{frrvar}.

\bigfig{fig:rdd-output}{insert/output}{100_rho5e13m-2_square_3200nm_RDD_d5e-5nm-2_edge_S0_PBC1_output}{Results of an X-ray diffraction simulation conducted on a sample of \gls{rdd}}

The example shown in \figref{fig:rdd-output} can be exported with the following script.

\pyscript{insert/output}{plot.py}

\bigskip

For the rest it is defined the orthonormal basis \( \left( \ux, \uy, \uz \right) \) as follow.
\uz \ gives the direction of the dislocation line vectors \gls{vecl}.
\ux \ gives the direction of the vector associated with the Fourier variable \gls{frrvar}.
And \( \uy = \uz \times \ux \).

\subsection{Fit models}

To determine the density \gls{dst} and the outer cut-off radius \gls{cutrad}, the theoretical models at our disposal are fitted to the simulation results.
These models are presented here.
It is first defined some quantities that will be common to the models.
\( k \) can be compared to the directing coefficient of the linear part in \gls{r2} in \figref{fig:rdd-output}.
\( \psi \) is the angle between \gls{vecg} and \gls{vecl}.
\gls{pssrat} is the \glsdesc{pssrat}.
\( \gamma \) is the angle between the projections of \gls{vecg} and \gls{vecb} in the plane orthogonal to \gls{vecl}.
Finally \gls{confac} is the contrast factor defined for edge and screw dislocations in elastically isotropic crystals.

\begin{equation}
  k =
    \frac{\pi}{2} | \gls{hmc} \gls{vecg} |^2 | \gls{vecb} |^2 \gls{confac} \gls{dst}
\end{equation}

\begin{equation}
  \psi =
    \angle \left( \gls{vecg}, \gls{vecl} \right)
\end{equation}

\begin{equation}
  \gamma = \angle \left(
    \gls{vecg} - \left( \gls{vecg} \cdot \uz \right) \uz,
    \gls{vecb} - \left( \gls{vecb} \cdot \uz \right) \uz
    \right)
\end{equation}

\begin{equation}\label{eq:contrast-factor}
  \gls{confac}_\mathrm{edge} =
    \begin{cases}
      \sin^4 \psi \frac{1}{8(1-\nu)^2} \left( 1 - 4 \nu + 8 \nu^2 + 4 (1 - 2 \nu) \cos^2 \gamma \right)
      & \text{for edge dislocations} \\[3mm]
      \sin^2 \psi \cos^2 \psi
      & \text{for screw dislocations}
    \end{cases}
\end{equation}

\subsubsection{\glsentrytext{guw1}}

Equation \eqref{eq:guw1} gives be the Fourier transform of diffraction vector harmonic \gls{hmc} as predicted in the \gls{guw1} \cite{GUW1988} \cite{SGL2001}.
Compared to the original proposal, \( \langle \gls{dst}^2 \rangle - \langle \gls{dst} \rangle^2 \) has been simplified to \( \delta \langle \gls{dst} \rangle^2 \) with \( \delta \) a coefficient quantifying the density fluctuation and \( \ln(\gls{frrvar}/R_1) \ln(\gls{frrvar}/R_2) \) has been simplified to \( (\ln(\gls{frrvar}/R_0))^2 \).

\begin{equation}
  \define{
    \gls{guw1}_{\gls{hmc}}
  }{
    \gls{setR}^*
  }{
    \gls{setR}
  }{
    \gls{frrvar}
  }{
    \exp\left( \gls{frrvar}^2 D_{\gls{hmc}}(\gls{frrvar}) \right)
  }
  \label{eq:guw1}
\end{equation}

\begin{equation}
  D_{\gls{hmc}}(\gls{frrvar}) =
    k \left( \ln \left( \frac{\gls{frrvar}}{\gls{cutrad}} \right) + \frac{\delta}{2} k \gls{frrvar}^2 \left( \ln\left( \frac{\gls{frrvar}}{R_0} \right) \right)^2 \right)
\end{equation}

\subsubsection{\glsentrytext{guw2}}

Equation \eqref{eq:guw2} gives be the Fourier transform of diffraction vector harmonic \gls{hmc} as predicted in the \gls{guw2} \cite{GUW1988}.

\begin{equation}
  \define{
    \gls{guw2}_{\gls{hmc}}
  }{
    \gls{setR}^*
  }{
    \gls{setR}
  }{
    \gls{frrvar}
  }{
    \exp\left( \gls{frrvar}^2 D_{\gls{hmc}}(\gls{frrvar}) \right)
  }
  \label{eq:guw2}
\end{equation}

\begin{equation}
  D_{\gls{hmc}}(\gls{frrvar}) =
    k \left( \ln \left( \gls{frrvar} \right) - \ln\left( \left( \gls{hmc} \gls{vecg} \cdot \gls{vecb} \right) \gls{cutrad} \right) \right)
\end{equation}

\subsubsection{\glsentrytext{w1}}

Equation \eqref{eq:w1} gives be the Fourier transform of diffraction vector harmonic \gls{hmc} as predicted in the \gls{w1} \cite{W1970}.

\begin{equation}
  \define{
    \gls{w1}_{\gls{hmc}}
  }{
    \gls{setR}^*
  }{
    \gls{setR}
  }{
    \gls{frrvar}
  }{
    W_{\gls{hmc}}^S(\gls{frrvar}) W_{\gls{hmc}}^D(\gls{frrvar})
  }
  \label{eq:w1}
\end{equation}

\begin{align}
  W_{\gls{hmc}}^S(\gls{frrvar}) &= 1 \\
  W_{\gls{hmc}}^D(\gls{frrvar}) &= \exp\left( \gls{frrvar}^2 D_{\gls{hmc}}(\gls{frrvar}) \right)
\end{align}

\begin{equation}
  D_{\gls{hmc}}(\gls{frrvar}) =
    - k f \left( \frac{1}{2e^\frac{1}{4}} \frac{\gls{frrvar}}{\gls{cutrad}} \right)
\end{equation}

\medskip

\begin{equation}
  f(\eta) =
    \begin{cases}
      - \ln(\eta) + \frac{7}{4} - \ln(2) + \frac{512}{90 \pi} \eta^{-1} \\[5mm]
      + \frac{2}{\pi} \left( 1 - \frac{1}{4}\eta^{-2} \right) \int_0^\eta \frac{\arcsin(V)}{V} dV \\[5mm]
      - \frac{1}{\pi} \left( \frac{769}{180} \eta^{-1} + \frac{41}{90} \eta + \frac{2}{90}\eta^3 \right) \sqrt{ 1 - \eta^2 }
      & \forall \eta \leq 1 \\[1cm]
      \frac{512}{90 \pi} \eta^{-1} - \left( \frac{11}{24} + \frac{1}{4} \ln(2) \eta \right) \eta^{-2}
      & \forall \eta > 1
    \end{cases}
\end{equation}

\subsubsection{\glsentrytext{w2}}

Equation \eqref{eq:w2} gives be the Fourier transform of diffraction vector harmonic \gls{hmc} as predicted in the \gls{w2} proposed by \textcite{KD2001}.

\begin{equation}
  \define{
    \gls{w2}_{\gls{hmc}}
  }{
    \gls{setR}^*
  }{
    \gls{setR}
  }{
    \gls{frrvar}
  }{
    \exp\left( \gls{frrvar}^2 D_{\gls{hmc}}(\gls{frrvar}) \right)
  }
  \label{eq:w2}
\end{equation}

\begin{equation}
  D_{\gls{hmc}}(\gls{frrvar}) =
    k \left( \ln \left( \gls{frrvar} \right) - \ln\left( \left( \gls{hmc} \gls{vecg} \cdot \gls{vecb} \right) \gls{cutrad} \right) - 2 \ln(2) +\frac{1}{3} + \ln\left(|\sin(\psi) \gls{vecg} \cdot \gls{vecb}| \right) \right)
\end{equation}

\subsection{Fits}

For each model, the fits are performed for each harmonic \gls{hmc} and for several maximum values of \gls{frrvar}.
The Nelder–Mead method is used to find the minimum of the objective function.
\figref{fig:fit:example:rdd} show a fit for a given \gls{hmc} and a given maximum value of \gls{frrvar}.

\bigfig{fig:fit:example:rdd}{insert/fits/100_rho5e13m-2_square_3200nm_RDD_d5e-5nm-2_edge_S0_PBC1_output_analysis/fits_plot_GUW2}{j1_118nm}{Fit of \gls{guw2} over the harmonic 1 of the computed Fourier Transform of \gls{rdd}}

\subsubsection{Filters}

It has been developed two ways of choosing the maximum value of the maximum values of \gls{frrvar}.
The first one, named \gls{f1}, aims at keeping only the relevant results and locates the point where \gls{frrtrf} becomes noise (i.e. when it is no longer decreasing).
The second filter, named \gls{f2}, marks the end of the linearity zone of \( \ln \left( \gls{frrtrf}(\gls{frrvar}) \right) / \gls{frrvar}^2 \) as a function of \( \ln(\gls{frrvar}) \).
Given the application assumptions of the models, the filter \gls{f1} is applied with \gls{w1} and \gls{f2} is applied with \gls{guw2} and \gls{w2}.
Indeed \gls{guw2} and \gls{w2} are asymptotic models and \gls{w1} is supposed to describe the complete line profile for \gls{rrdd}.
The purpose of these filters is therefore to apply the models to the intervals for which they are intended.

\medfig{fig:filter-1}{insert/filters}{f1}{Example of filtering with \gls{f1}}%
\medfig{fig:filter-2}{insert/filters}{f2}{Example of filtering with \gls{f2}}%

\bigskip

\gls{f1} stops at index \( i \) when equation \eqref{eq:filter-1} is not verified.
In the same way \gls{f2} stops at index \( i \) when equation \eqref{eq:filter-2} is not verified.

\begin{equation}\label{eq:filter-1}
y_{i+1} < y_i
\end{equation}

\begin{equation}\label{eq:filter-2}
\left| \frac{\hat{\gls{stddev}}_{\text{Linear regression of } \{ y_i \}_{j<i}}}{\max \left( \{ y_i \}_{j<i} \right) - \min \left( \{ y_i \}_{j<i} \right)} \right| < 0.01
\end{equation}

\subsubsection{Residuals}

The models \gls{guw2} and \gls{w2} are fitted within the representation \( \ln \left( \gls{frrtrf}(\gls{frrvar}) \right)/L^2 \) as a function of \gls{frrvar} and the model \gls{w1} is fitted within the representation \( \gls{frrtrf}(\gls{frrvar}) \) as a function of \( \ln(\gls{frrvar}) \). Thus the residuals are defined in \eqref{eq:fit-residual-1} for \gls{w1} and in \eqref{eq:fit-residual-2} for \gls{guw2} and \gls{w2}.

\begin{align}
\hat{\epsilon}_i &=
\gls{frrtrf}(\gls{frrvar}_i) - \gls{frrtrf}^{M}(\gls{frrvar}_i) \label{eq:fit-residual-1} \\
\hat{\epsilon}_i &=
\frac{\ln\left(\gls{frrtrf}(\gls{frrvar}_i)\right)}{\gls{frrvar}^2} -
\frac{\ln\left(\gls{frrtrf}^{MOD}(\gls{frrvar}_i)\right)}{\gls{frrvar}^2} \label{eq:fit-residual-2}
\end{align}

\subsubsection{Weighting}
The application of the logarithmic scale to the abscissa axis in the representation of \( \ln \left( \gls{frrtrf}(\gls{frrvar}) \right) / \gls{frrvar}^2 \) as a function of \( \ln(\gls{frrvar}) \) induces a rapprochement of the measurement points with \gls{frrvar} increasing.
Here is proposed a weighting of each point according to the distance between it and its neighbors.
With this weighting, the fits fit better to the linear parts in the latter representation.

\begin{align}
w^{\gls{w1}}_i &= 1 \\
w^{\gls{guw2}}_i = w^{\gls{w2}}_i &=
n
\frac{
\ln \left(\gls{frrvar}_{i + 1} \right) - \ln \left(\gls{frrvar}_i\right)
}{
\ln \left(\gls{frrvar}_{n + 1} \right) - \ln \left(\gls{frrvar}_1\right)
} =
\frac{n}{\ln \left( n + 1 \right)} \ln \left( \frac{i+1}{i} \right)
\end{align}

\subsubsection{Standard fit error}
The objective function to be minimized for a given model \( M \), a harmonic \gls{hmc} and a maximum value \( \gls{frrvar}_n \) of \gls{frrvar} is the standard error of the fit defined in equation \eqref{eq:fit-standard-error}.

\begin{equation}\label{eq:fit-standard-error}
\hat{\gls{stddev}} \left( \gls{frrtrf}^M \right) = \sqrt{\sum_{i=1}^n \frac{w^M_i\hat{\epsilon}_i^2}{n-2}}
\end{equation}

\subsection{Synthesis}\label{sec:synthesis}

Below is an overview of the parameters used for the generation of the samples of distributions.

\begin{tcolorbox}
\begin{verbatim}
5e13m-2:
5.000e-05m-2 square 3200nm RDD (d=5e-5nm-2) S=0
5.000e-05m-2 square 3200nm RRDD-E (d=5e-5nm-2 s= 200nm) S=0
5.000e-05m-2 square 3200nm RRDD-E (d=5e-5nm-2 s= 400nm) S=0
5.000e-05m-2 square 3200nm RRDD-R (d=5e-5nm-2 s= 200nm) S=0
5.000e-05m-2 square 3200nm RRDD-R (d=5e-5nm-2 s= 400nm) S=0
5.000e-05m-2 square 3200nm RCDD-E (d=5e-5nm-2 s= 800nm t= 84nm) S=0
5.000e-05m-2 square 3200nm RCDD-R (d=5e-5nm-2 s= 800nm t= 84nm) S=0
5.000e-05m-2 square 3200nm RCDD-D (d=5e-5nm-2 s= 800nm t= 84nm l= 71nm) S=0
5.000e-05m-2 square 3200nm RCDD-D (d=5e-5nm-2 s= 800nm t= 84nm l=141nm) S=0
5.000e-05m-2 square 3200nm RCDD-D (d=5e-5nm-2 s= 800nm t= 84nm l=283nm) S=0

5e14m-2:
5.000e-04m-2 square 3200nm RDD (d=5e-4nm-2) S=0
5.000e-04m-2 square 3200nm RRDD-E (d=5e-4nm-2 s= 200nm) S=0
5.000e-04m-2 square 3200nm RRDD-E (d=5e-4nm-2 s= 400nm) S=0
5.000e-04m-2 square 3200nm RRDD-R (d=5e-4nm-2 s= 200nm) S=0
5.000e-04m-2 square 3200nm RRDD-R (d=5e-4nm-2 s= 400nm) S=0
5.000e-04m-2 square 3200nm RCDD-E (d=5e-4nm-2 s= 800nm t= 84nm) S=0
5.000e-04m-2 square 3200nm RCDD-R (d=5e-4nm-2 s= 800nm t= 84nm) S=0
5.000e-04m-2 square 3200nm RCDD-D (d=5e-4nm-2 s= 800nm t= 84nm l= 22nm) S=0
5.000e-04m-2 square 3200nm RCDD-D (d=5e-4nm-2 s= 800nm t= 84nm l= 45nm) S=0
5.000e-04m-2 square 3200nm RCDD-D (d=5e-4nm-2 s= 800nm t= 84nm l= 89nm) S=0

5e15m-2:
5.000e-03m-2 square 3200nm RDD (d=5e-3nm-2) S=0
5.000e-03m-2 square 3200nm RRDD-E (d=5e-3nm-2 s= 200nm) S=0
5.000e-03m-2 square 3200nm RRDD-E (d=5e-3nm-2 s= 400nm) S=0
5.000e-03m-2 square 3200nm RRDD-R (d=5e-3nm-2 s= 200nm) S=0
5.000e-03m-2 square 3200nm RRDD-R (d=5e-3nm-2 s= 400nm) S=0
5.000e-03m-2 square 3200nm RCDD-E (d=5e-3nm-2 s= 800nm t= 84nm) S=0
5.000e-03m-2 square 3200nm RCDD-R (d=5e-3nm-2 s= 800nm t= 84nm) S=0
5.000e-03m-2 square 3200nm RCDD-D (d=5e-3nm-2 s= 800nm t= 84nm l=  7nm) S=0
5.000e-03m-2 square 3200nm RCDD-D (d=5e-3nm-2 s= 800nm t= 84nm l= 14nm) S=0
5.000e-03m-2 square 3200nm RCDD-D (d=5e-3nm-2 s= 800nm t= 84nm l= 28nm) S=0
\end{verbatim}
\end{tcolorbox}

\bigskip

These distribution models are illustrated in section \ref{sec:data}.
Here is proposed a synthesis of the results of the fits performed on the simulation outputs of the latter.
The objective was to gather in a single data the fits performed for several maximum values of \gls{frrvar} for a given distribution, model and harmonic.
To simplify the reading of the tables, the density values closest to the real densities have been colored.
Not all decimal places are displayed.

\bigskip

\newpage

{\renewcommand{\arraystretch}{1.6}

\subsubsection{Mean effective cut-off radius and relative mean density deviation}

\begin{center}
\begin{tabularx}{\linewidth}{|l|l|>{\raggedleft\arraybackslash}X|>{\raggedleft\arraybackslash}X|>{\raggedleft\arraybackslash}X|>{\raggedleft\arraybackslash}X|>{\raggedleft\arraybackslash}X|>{\raggedleft\arraybackslash}X|>{\raggedleft\arraybackslash}X|>{\raggedleft\arraybackslash}X|} 
\hline
\multirow{2}{*}{\centering{Distribution}} & \multicolumn{1}{c|}{\centering{$ \hat{\gls{expval}} \left( \delta \right) $}} & \multicolumn{4}{c|}{ $ \left. \left|\hat{\gls{expval}} \left( \gls{dst}^{\mathrm{FIT}} \right)-\gls{dst}\right| \right/ \gls{dst} $} & \multicolumn{4}{c|}{$ \hat{\gls{expval}} \left( \gls{cutrad}^{\mathrm{FIT}} \right) $ (nm)} \\
\cline{2-10}
 & \multicolumn{2}{c|}{\gls{guw1}} & \multicolumn{1}{c|}{\gls{guw2}} & \multicolumn{1}{c|}{\gls{w1}} & \multicolumn{1}{c|}{\gls{w2}} & \multicolumn{1}{c|}{\gls{guw1}} & \multicolumn{1}{c|}{\gls{guw2}} & \multicolumn{1}{c|}{\gls{w1}} & \multicolumn{1}{c|}{\gls{w2}} \\
\hline \hline 
\hyperref[RRDD-E-5e13-A]{\texttt{\verb|RRDD-E-5e13-A|}} & \( 0.916 \) & \( 0.281 \) & \cellcolor{Mines} \textcolor{white}{\( 0.063 \)} & \( 0.135 \) & \cellcolor{Mines} \textcolor{white}{\( 0.063 \)} & \( 18457 \) & \( 766 \) & \( 228 \) & \( 244 \) \\
\hyperref[RRDD-E-5e13-B]{\texttt{\verb|RRDD-E-5e13-B|}} & \( 0.000 \) & \( 0.118 \) & \cellcolor{Mines} \textcolor{white}{\( 0.041 \)} & \( 0.112 \) & \cellcolor{Mines} \textcolor{white}{\( 0.041 \)} & \( 1881 \) & \( 1417 \) & \( 409 \) & \( 451 \) \\
\hyperref[RRDD-E-5e14-A]{\texttt{\verb|RRDD-E-5e14-A|}} & \( 0.341 \) & \( 0.057 \) & \cellcolor{Mines} \textcolor{white}{\( 0.035 \)} & \( 0.072 \) & \cellcolor{Mines} \textcolor{white}{\( 0.035 \)} & \( 762 \) & \( 683 \) & \( 173 \) & \( 218 \) \\
\hyperref[RRDD-E-5e14-B]{\texttt{\verb|RRDD-E-5e14-B|}} & \( 0.457 \) & \( 0.060 \) & \cellcolor{Mines} \textcolor{white}{\( 0.015 \)} & \( 0.038 \) & \cellcolor{Mines} \textcolor{white}{\( 0.015 \)} & \( 1744 \) & \( 1277 \) & \( 314 \) & \( 407 \) \\
\hyperref[RRDD-E-5e15-A]{\texttt{\verb|RRDD-E-5e15-A|}} & \( 0.000 \) & \( 0.012 \) & \cellcolor{Mines} \textcolor{white}{\( 0.005 \)} & \( 0.011 \) & \cellcolor{Mines} \textcolor{white}{\( 0.005 \)} & \( 611 \) & \( 591 \) & \( 136 \) & \( 188 \) \\
\hyperref[RRDD-E-5e15-B]{\texttt{\verb|RRDD-E-5e15-B|}} & \( 0.000 \) & \( 0.018 \) & \cellcolor{Mines} \textcolor{white}{\( 0.011 \)} & \( 0.018 \) & \cellcolor{Mines} \textcolor{white}{\( 0.011 \)} & \( 1299 \) & \( 1259 \) & \( 290 \) & \( 401 \) \\
\hline
\hyperref[RRDD-R-5e13-A]{\texttt{\verb|RRDD-R-5e13-A|}} & \( 0.730 \) & \( 0.142 \) & \cellcolor{Mines} \textcolor{white}{\( 0.058 \)} & \( 0.164 \) & \cellcolor{Mines} \textcolor{white}{\( 0.058 \)} & \( 13281 \) & \( 7498 \) & \( 3345 \) & \( 2388 \) \\
\hyperref[RRDD-R-5e13-B]{\texttt{\verb|RRDD-R-5e13-B|}} & \( 1.855 \) & \( 0.147 \) & \cellcolor{Mines} \textcolor{white}{\( 0.023 \)} & \( 0.073 \) & \cellcolor{Mines} \textcolor{white}{\( 0.023 \)} & \( 30852 \) & \( 6182 \) & \( 1815 \) & \( 1969 \) \\
\hyperref[RRDD-R-5e14-A]{\texttt{\verb|RRDD-R-5e14-A|}} & \( 1.118 \) & \( 0.071 \) & \cellcolor{Mines} \textcolor{white}{\( 0.032 \)} & \( 0.086 \) & \cellcolor{Mines} \textcolor{white}{\( 0.032 \)} & \( 9522 \) & \( 6794 \) & \( 2199 \) & \( 2164 \) \\
\hyperref[RRDD-R-5e14-B]{\texttt{\verb|RRDD-R-5e14-B|}} & \( 0.768 \) & \cellcolor{Mines} \textcolor{white}{\( 0.002 \)} & \( 0.008 \) & \( 0.012 \) & \( 0.008 \) & \( 5354 \) & \( 5533 \) & \( 1269 \) & \( 1762 \) \\
\hyperref[RRDD-R-5e15-A]{\texttt{\verb|RRDD-R-5e15-A|}} & \( 0.000 \) & \( 0.033 \) & \cellcolor{Mines} \textcolor{white}{\( 0.026 \)} & \( 0.033 \) & \cellcolor{Mines} \textcolor{white}{\( 0.026 \)} & \( 6817 \) & \( 6563 \) & \( 1521 \) & \( 2090 \) \\
\hyperref[RRDD-R-5e15-B]{\texttt{\verb|RRDD-R-5e15-B|}} & \( 0.161 \) & \( 0.019 \) & \cellcolor{Mines} \textcolor{white}{\( 0.001 \)} & \( 0.003 \) & \cellcolor{Mines} \textcolor{white}{\( 0.001 \)} & \( 5870 \) & \( 4974 \) & \( 1075 \) & \( 1584 \) \\
\hline
\hyperref[RDD-5e13]{\texttt{\verb|RDD-5e13|}} & \( 0.000 \) & \( 0.031 \) & \cellcolor{Mines} \textcolor{white}{\( 0.013 \)} & \( 0.031 \) & \cellcolor{Mines} \textcolor{white}{\( 0.013 \)} & \( 6026 \) & \( 5549 \) & \( 1342 \) & \( 1768 \) \\
\hyperref[RDD-5e14]{\texttt{\verb|RDD-5e14|}} & \( 1.003 \) & \( 0.086 \) & \cellcolor{Mines} \textcolor{white}{\( 0.011 \)} & \( 0.031 \) & \cellcolor{Mines} \textcolor{white}{\( 0.011 \)} & \( 17355 \) & \( 5970 \) & \( 1509 \) & \( 1901 \) \\
\hyperref[RDD-5e15]{\texttt{\verb|RDD-5e15|}} & \( 0.000 \) & \( 0.001 \) & \( 0.002 \) & \cellcolor{Mines} \textcolor{white}{\( 0.001 \)} & \( 0.002 \) & \( 5223 \) & \( 5278 \) & \( 1159 \) & \( 1681 \) \\
\hline
\hyperref[RCDD-D-5e13-A]{\texttt{\verb|RCDD-D-5e13-A|}} & \( 1.243 \) & \cellcolor{Mines} \textcolor{white}{\( 0.166 \)} & \( 0.191 \) & \( \rightarrow \infty \) & \( 0.191 \) & \( 463 \) & \( 223 \) & \( 5 \) & \( 71 \) \\
\hyperref[RCDD-D-5e13-B]{\texttt{\verb|RCDD-D-5e13-B|}} & \( 0.815 \) & \cellcolor{Mines} \textcolor{white}{\( 0.060 \)} & \( 0.159 \) & \( 1.555 \) & \( 0.159 \) & \( 643 \) & \( 461 \) & \( 30 \) & \( 147 \) \\
\hyperref[RCDD-D-5e13-C]{\texttt{\verb|RCDD-D-5e13-C|}} & \( 0.500 \) & \cellcolor{Mines} \textcolor{white}{\( 0.022 \)} & \( 0.084 \) & \( 0.508 \) & \( 0.084 \) & \( 1668 \) & \( 1148 \) & \( 112 \) & \( 366 \) \\
\hyperref[RCDD-D-5e14-A]{\texttt{\verb|RCDD-D-5e14-A|}} & \( 1.666 \) & \( 0.434 \) & \cellcolor{Mines} \textcolor{white}{\( 0.313 \)} & \( \rightarrow \infty \) & \cellcolor{Mines} \textcolor{white}{\( 0.313 \)} & \( 191 \) & \( 52 \) & \( 0 \) & \( 16 \) \\
\hyperref[RCDD-D-5e14-B]{\texttt{\verb|RCDD-D-5e14-B|}} & \( 1.379 \) & \cellcolor{Mines} \textcolor{white}{\( 0.270 \)} & \( 0.398 \) & \( \rightarrow \infty \) & \( 0.398 \) & \( 210 \) & \( 71 \) & \( 1 \) & \( 23 \) \\
\hyperref[RCDD-D-5e14-C]{\texttt{\verb|RCDD-D-5e14-C|}} & \( 1.186 \) & \cellcolor{Mines} \textcolor{white}{\( 0.031 \)} & \( 0.397 \) & \( \rightarrow \infty \) & \( 0.397 \) & \( 273 \) & \( 120 \) & \( 1 \) & \( 38 \) \\
\hyperref[RCDD-D-5e15-A]{\texttt{\verb|RCDD-D-5e15-A|}} & \( 1.562 \) & \( 0.644 \) & \cellcolor{Mines} \textcolor{white}{\( 0.455 \)} & \( \rightarrow \infty \) & \cellcolor{Mines} \textcolor{white}{\( 0.455 \)} & \( 88 \) & \( 13 \) & \( 0 \) & \( 4 \) \\
\hyperref[RCDD-D-5e15-B]{\texttt{\verb|RCDD-D-5e15-B|}} & \( 1.426 \) & \cellcolor{Mines} \textcolor{white}{\( 0.574 \)} & \( 0.741 \) & \( \rightarrow \infty \) & \( 0.741 \) & \( 91 \) & \( 13 \) & \( 0 \) & \( 4 \) \\
\hyperref[RCDD-D-5e15-C]{\texttt{\verb|RCDD-D-5e15-C|}} & \( 1.015 \) & \cellcolor{Mines} \textcolor{white}{\( 0.458 \)} & \( 1.003 \) & \( \rightarrow \infty \) & \( 1.003 \) & \( 92 \) & \( 14 \) & \( 0 \) & \( 4 \) \\
\hline
\hyperref[RCDD-E-5e13]{\texttt{\verb|RCDD-E-5e13|}} & \( 1.536 \) & \cellcolor{Mines} \textcolor{white}{\( 0.163 \)} & \( 0.261 \) & \( 0.253 \) & \( 0.261 \) & \( 1184 \) & \( 712 \) & \( 153 \) & \( 227 \) \\
\hyperref[RCDD-E-5e14]{\texttt{\verb|RCDD-E-5e14|}} & \( 2.999 \) & \cellcolor{Mines} \textcolor{white}{\( 0.494 \)} & \( 0.592 \) & \( 0.777 \) & \( 0.592 \) & \( 1140 \) & \( 207 \) & \( 31 \) & \( 66 \) \\
\hyperref[RCDD-E-5e15]{\texttt{\verb|RCDD-E-5e15|}} & \( 0.772 \) & \cellcolor{Mines} \textcolor{white}{\( 0.868 \)} & \( 1.466 \) & \( 1.092 \) & \( 1.466 \) & \( 57 \) & \( 26 \) & \( 8 \) & \( 8 \) \\
\hline
\hyperref[RCDD-R-5e13]{\texttt{\verb|RCDD-R-5e13|}} & \( 0.002 \) & \cellcolor{Mines} \textcolor{white}{\( 0.249 \)} & \( 0.281 \) & \( 0.288 \) & \( 0.281 \) & \( 1158 \) & \( 1138 \) & \( 234 \) & \( 363 \) \\
\hyperref[RCDD-R-5e14]{\texttt{\verb|RCDD-R-5e14|}} & \( 3.712 \) & \cellcolor{Mines} \textcolor{white}{\( 0.453 \)} & \( 0.592 \) & \( 0.757 \) & \( 0.592 \) & \( 2865 \) & \( 316 \) & \( 48 \) & \( 101 \) \\
\hyperref[RCDD-R-5e15]{\texttt{\verb|RCDD-R-5e15|}} & \( 1.103 \) & \cellcolor{Mines} \textcolor{white}{\( 1.164 \)} & \( 1.503 \) & \( 1.285 \) & \( 1.503 \) & \( 61 \) & \( 33 \) & \( 9 \) & \( 10 \) \\
\hline
\end{tabularx}
\end{center}

\captionof{table}{Mean effective cut-off radius and relative mean density deviation for harmonic \( j = 1 \) of \gls{guw1}, \gls{guw2}, \gls{w1} and \gls{w2} models fitted on the X-ray simulation output performed on distributions generated in a square \gls{roi} of 3200 nm side length with \gls{pbc}1.}

\newpage

\begin{center}
\begin{tabularx}{\linewidth}{|l|l|>{\raggedleft\arraybackslash}X|>{\raggedleft\arraybackslash}X|>{\raggedleft\arraybackslash}X|>{\raggedleft\arraybackslash}X|>{\raggedleft\arraybackslash}X|>{\raggedleft\arraybackslash}X|>{\raggedleft\arraybackslash}X|>{\raggedleft\arraybackslash}X|} 
\hline
\multirow{2}{*}{\centering{Distribution}} & \multicolumn{1}{c|}{\centering{$ \hat{\gls{expval}} \left( \delta \right) $}} & \multicolumn{4}{c|}{ $ \left. \left|\hat{\gls{expval}} \left( \gls{dst}^{\mathrm{FIT}} \right)-\gls{dst}\right| \right/ \gls{dst} $} & \multicolumn{4}{c|}{$ \hat{\gls{expval}} \left( \gls{cutrad}^{\mathrm{FIT}} \right) $ (nm)} \\
\cline{2-10}
 & \multicolumn{2}{c|}{\gls{guw1}} & \multicolumn{1}{c|}{\gls{guw2}} & \multicolumn{1}{c|}{\gls{w1}} & \multicolumn{1}{c|}{\gls{w2}} & \multicolumn{1}{c|}{\gls{guw1}} & \multicolumn{1}{c|}{\gls{guw2}} & \multicolumn{1}{c|}{\gls{w1}} & \multicolumn{1}{c|}{\gls{w2}} \\
\hline \hline 
\hyperref[RRDD-E-5e13-A]{\texttt{\verb|RRDD-E-5e13-A|}} & \( 1.752 \) & \( 0.238 \) & \cellcolor{Mines} \textcolor{white}{\( 0.110 \)} & \( 0.185 \) & \cellcolor{Mines} \textcolor{white}{\( 0.110 \)} & \( 4070 \) & \( 205 \) & \( 116 \) & \( 131 \) \\
\hyperref[RRDD-E-5e13-B]{\texttt{\verb|RRDD-E-5e13-B|}} & \( 0.000 \) & \( 0.122 \) & \cellcolor{Mines} \textcolor{white}{\( 0.059 \)} & \( 0.112 \) & \cellcolor{Mines} \textcolor{white}{\( 0.059 \)} & \( 853 \) & \( 345 \) & \( 183 \) & \( 220 \) \\
\hyperref[RRDD-E-5e14-A]{\texttt{\verb|RRDD-E-5e14-A|}} & \( 0.413 \) & \( 0.053 \) & \cellcolor{Mines} \textcolor{white}{\( 0.048 \)} & \( 0.068 \) & \cellcolor{Mines} \textcolor{white}{\( 0.048 \)} & \( 342 \) & \( 163 \) & \( 78 \) & \( 104 \) \\
\hyperref[RRDD-E-5e14-B]{\texttt{\verb|RRDD-E-5e14-B|}} & \( 0.595 \) & \( 0.065 \) & \cellcolor{Mines} \textcolor{white}{\( 0.023 \)} & \( 0.042 \) & \cellcolor{Mines} \textcolor{white}{\( 0.023 \)} & \( 802 \) & \( 304 \) & \( 146 \) & \( 193 \) \\
\hyperref[RRDD-E-5e15-A]{\texttt{\verb|RRDD-E-5e15-A|}} & \( 0.000 \) & \( 0.014 \) & \cellcolor{Mines} \textcolor{white}{\( 0.001 \)} & \( 0.008 \) & \cellcolor{Mines} \textcolor{white}{\( 0.001 \)} & \( 283 \) & \( 133 \) & \( 61 \) & \( 85 \) \\
\hyperref[RRDD-E-5e15-B]{\texttt{\verb|RRDD-E-5e15-B|}} & \( 0.000 \) & \( 0.021 \) & \( 0.028 \) & \cellcolor{Mines} \textcolor{white}{\( 0.021 \)} & \( 0.028 \) & \( 605 \) & \( 315 \) & \( 135 \) & \( 201 \) \\
\hline
\hyperref[RRDD-R-5e13-A]{\texttt{\verb|RRDD-R-5e13-A|}} & \( 1.439 \) & \( 0.143 \) & \cellcolor{Mines} \textcolor{white}{\( 0.106 \)} & \( 0.159 \) & \cellcolor{Mines} \textcolor{white}{\( 0.106 \)} & \( 6354 \) & \( 2273 \) & \( 1400 \) & \( 1448 \) \\
\hyperref[RRDD-R-5e13-B]{\texttt{\verb|RRDD-R-5e13-B|}} & \( 1.314 \) & \( 0.104 \) & \cellcolor{Mines} \textcolor{white}{\( 0.033 \)} & \( 0.060 \) & \cellcolor{Mines} \textcolor{white}{\( 0.033 \)} & \( 15532 \) & \( 1488 \) & \( 757 \) & \( 948 \) \\
\hyperref[RRDD-R-5e14-A]{\texttt{\verb|RRDD-R-5e14-A|}} & \( 1.306 \) & \( 0.075 \) & \cellcolor{Mines} \textcolor{white}{\( 0.051 \)} & \( 0.091 \) & \cellcolor{Mines} \textcolor{white}{\( 0.051 \)} & \( 4435 \) & \( 1753 \) & \( 1017 \) & \( 1117 \) \\
\hyperref[RRDD-R-5e14-B]{\texttt{\verb|RRDD-R-5e14-B|}} & \( 0.795 \) & \cellcolor{Mines} \textcolor{white}{\( 0.006 \)} & \( 0.007 \) & \( 0.012 \) & \( 0.007 \) & \( 2633 \) & \( 1270 \) & \( 582 \) & \( 809 \) \\
\hyperref[RRDD-R-5e15-A]{\texttt{\verb|RRDD-R-5e15-A|}} & \( 0.000 \) & \cellcolor{Mines} \textcolor{white}{\( 0.046 \)} & \( 0.061 \) & \cellcolor{Mines} \textcolor{white}{\( 0.046 \)} & \( 0.061 \) & \( 3436 \) & \( 1941 \) & \( 767 \) & \( 1237 \) \\
\hyperref[RRDD-R-5e15-B]{\texttt{\verb|RRDD-R-5e15-B|}} & \( 1.636 \) & \( 0.214 \) & \cellcolor{Mines} \textcolor{white}{\( 0.001 \)} & \( 0.015 \) & \cellcolor{Mines} \textcolor{white}{\( 0.001 \)} & \( 52697 \) & \( 1152 \) & \( 456 \) & \( 734 \) \\
\hline
\hyperref[RDD-5e13]{\texttt{\verb|RDD-5e13|}} & \( 0.000 \) & \( 0.030 \) & \cellcolor{Mines} \textcolor{white}{\( 0.020 \)} & \( 0.029 \) & \cellcolor{Mines} \textcolor{white}{\( 0.020 \)} & \( 2728 \) & \( 1316 \) & \( 607 \) & \( 838 \) \\
\hyperref[RDD-5e14]{\texttt{\verb|RDD-5e14|}} & \( 2.571 \) & \( 0.166 \) & \cellcolor{Mines} \textcolor{white}{\( 0.017 \)} & \( 0.038 \) & \cellcolor{Mines} \textcolor{white}{\( 0.017 \)} & \( 16916 \) & \( 1420 \) & \( 718 \) & \( 904 \) \\
\hyperref[RDD-5e15]{\texttt{\verb|RDD-5e15|}} & \( 0.000 \) & \( 0.002 \) & \( 0.030 \) & \cellcolor{Mines} \textcolor{white}{\( 0.002 \)} & \( 0.030 \) & \( 2398 \) & \( 1491 \) & \( 535 \) & \( 950 \) \\
\hline
\hyperref[RCDD-D-5e13-A]{\texttt{\verb|RCDD-D-5e13-A|}} & \( 0.900 \) & \( 0.430 \) & \cellcolor{Mines} \textcolor{white}{\( 0.221 \)} & \( \rightarrow \infty \) & \cellcolor{Mines} \textcolor{white}{\( 0.221 \)} & \( 410 \) & \( 51 \) & \( 1 \) & \( 33 \) \\
\hyperref[RCDD-D-5e13-B]{\texttt{\verb|RCDD-D-5e13-B|}} & \( 0.628 \) & \cellcolor{Mines} \textcolor{white}{\( 0.129 \)} & \( 0.273 \) & \( 1.348 \) & \( 0.273 \) & \( 474 \) & \( 86 \) & \( 15 \) & \( 55 \) \\
\hyperref[RCDD-D-5e13-C]{\texttt{\verb|RCDD-D-5e13-C|}} & \( 0.386 \) & \cellcolor{Mines} \textcolor{white}{\( 0.079 \)} & \( 0.176 \) & \( 0.450 \) & \( 0.176 \) & \( 640 \) & \( 201 \) & \( 55 \) & \( 128 \) \\
\hyperref[RCDD-D-5e14-A]{\texttt{\verb|RCDD-D-5e14-A|}} & \( 2.001 \) & \( 0.687 \) & \cellcolor{Mines} \textcolor{white}{\( 0.352 \)} & \( \rightarrow \infty \) & \cellcolor{Mines} \textcolor{white}{\( 0.352 \)} & \( 325 \) & \( 12 \) & \( 0 \) & \( 8 \) \\
\hyperref[RCDD-D-5e14-B]{\texttt{\verb|RCDD-D-5e14-B|}} & \( 0.769 \) & \cellcolor{Mines} \textcolor{white}{\( 0.539 \)} & \( 0.572 \) & \( \rightarrow \infty \) & \( 0.572 \) & \( 188 \) & \( 14 \) & \( 0 \) & \( 9 \) \\
\hyperref[RCDD-D-5e14-C]{\texttt{\verb|RCDD-D-5e14-C|}} & \( 0.892 \) & \cellcolor{Mines} \textcolor{white}{\( 0.156 \)} & \( 0.702 \) & \( \rightarrow \infty \) & \( 0.702 \) & \( 156 \) & \( 18 \) & \( 0 \) & \( 12 \) \\
\hyperref[RCDD-D-5e15-A]{\texttt{\verb|RCDD-D-5e15-A|}} & \( 0.025 \) & \( 0.919 \) & \cellcolor{Mines} \textcolor{white}{\( 0.162 \)} & \( \rightarrow \infty \) & \cellcolor{Mines} \textcolor{white}{\( 0.162 \)} & \( 100 \) & \( 4 \) & \( 0 \) & \( 3 \) \\
\hyperref[RCDD-D-5e15-B]{\texttt{\verb|RCDD-D-5e15-B|}} & \( 0.989 \) & \( 0.787 \) & \cellcolor{Mines} \textcolor{white}{\( 0.454 \)} & \( \rightarrow \infty \) & \cellcolor{Mines} \textcolor{white}{\( 0.454 \)} & \( 90 \) & \( 4 \) & \( 0 \) & \( 3 \) \\
\hyperref[RCDD-D-5e15-C]{\texttt{\verb|RCDD-D-5e15-C|}} & \( 0.790 \) & \cellcolor{Mines} \textcolor{white}{\( 0.713 \)} & \( 0.725 \) & \( \rightarrow \infty \) & \( 0.725 \) & \( 94 \) & \( 4 \) & \( 0 \) & \( 3 \) \\
\hline
\hyperref[RCDD-E-5e13]{\texttt{\verb|RCDD-E-5e13|}} & \( 1.281 \) & \cellcolor{Mines} \textcolor{white}{\( 0.134 \)} & \( 0.307 \) & \( 0.186 \) & \( 0.307 \) & \( 577 \) & \( 155 \) & \( 87 \) & \( 99 \) \\
\hyperref[RCDD-E-5e14]{\texttt{\verb|RCDD-E-5e14|}} & \( 0.821 \) & \cellcolor{Mines} \textcolor{white}{\( 0.655 \)} & \( 0.873 \) & \( 0.714 \) & \( 0.873 \) & \( 102 \) & \( 32 \) & \( 17 \) & \( 20 \) \\
\hyperref[RCDD-E-5e15]{\texttt{\verb|RCDD-E-5e15|}} & \( 1.293 \) & \cellcolor{Mines} \textcolor{white}{\( 0.986 \)} & \( 1.119 \) & \( 1.005 \) & \( 1.119 \) & \( 28 \) & \( 9 \) & \( 4 \) & \( 6 \) \\
\hline
\hyperref[RCDD-R-5e13]{\texttt{\verb|RCDD-R-5e13|}} & \( 0.723 \) & \( 0.254 \) & \( 0.358 \) & \cellcolor{Mines} \textcolor{white}{\( 0.238 \)} & \( 0.358 \) & \( 659 \) & \( 226 \) & \( 130 \) & \( 144 \) \\
\hyperref[RCDD-R-5e14]{\texttt{\verb|RCDD-R-5e14|}} & \( 1.572 \) & \cellcolor{Mines} \textcolor{white}{\( 0.577 \)} & \( 0.867 \) & \( 0.713 \) & \( 0.867 \) & \( 258 \) & \( 46 \) & \( 25 \) & \( 29 \) \\
\hyperref[RCDD-R-5e15]{\texttt{\verb|RCDD-R-5e15|}} & \( 1.259 \) & \cellcolor{Mines} \textcolor{white}{\( 1.090 \)} & \( 1.194 \) & \( 1.182 \) & \( 1.194 \) & \( 33 \) & \( 11 \) & \( 5 \) & \( 7 \) \\
\hline
\end{tabularx}
\end{center}

\captionof{table}{Mean effective cut-off radius and relative mean density deviation for harmonic \( j = 2 \) of \gls{guw1}, \gls{guw2}, \gls{w1} and \gls{w2} models fitted on the X-ray simulation output performed on distributions generated in a square \gls{roi} of 3200 nm side length with \gls{pbc}1.}

\newpage



\medfig{fig:avg-RRDD-E-A}{load/plots}{avg-RRDD-E-A}{RRDD-E-A}%
\medfig{fig:avg-RRDD-E-B}{load/plots}{avg-RRDD-E-B}{RRDD-E-B}%

\medfig{fig:avg-RRDD-R-A}{load/plots}{avg-RRDD-R-A}{RRDD-R-A}%
\medfig{fig:avg-RRDD-R-B}{load/plots}{avg-RRDD-R-B}{RRDD-R-B}%

\medfig{fig:avg-RDD}{load/plots}{avg-RDD}{RDD}%
\medfig{fig:avg-RCDD-D-A}{load/plots}{avg-RCDD-D-A}{RCDD-D-A}%

\medfig{fig:avg-RCDD-D-B}{load/plots}{avg-RCDD-D-B}{RCDD-D-B}%
\medfig{fig:avg-RCDD-D-C}{load/plots}{avg-RCDD-D-C}{RCDD-D-C}%

\medfig{fig:avg-RCDD-E}{load/plots}{avg-RCDD-E}{RCDD-E}%
\medfig{fig:avg-RCDD-R}{load/plots}{avg-RCDD-R}{RCDD-R}%



\newpage

\subsubsection{Effective cut-off radius sandard deviation and relative density standard deviation}

\begin{center}
\begin{tabularx}{\linewidth}{|l|l|>{\raggedleft\arraybackslash}X|>{\raggedleft\arraybackslash}X|>{\raggedleft\arraybackslash}X|>{\raggedleft\arraybackslash}X|>{\raggedleft\arraybackslash}X|>{\raggedleft\arraybackslash}X|>{\raggedleft\arraybackslash}X|>{\raggedleft\arraybackslash}X|} 
\hline
\multirow{2}{*}{\centering{Distribution}} & \multicolumn{1}{c|}{\centering{$ \hat{\gls{stddev}} \left( \delta \right) $}} & \multicolumn{4}{c|}{ $ \left. \hat{\gls{stddev}} \left( \gls{dst}^{\mathrm{FIT}} \right) \right/ \gls{dst} $} & \multicolumn{4}{c|}{$ \hat{\gls{stddev}} \left( \gls{cutrad}^{\mathrm{FIT}} \right) $ (nm)} \\
\cline{2-10}
 & \multicolumn{2}{c|}{\gls{guw1}} & \multicolumn{1}{c|}{\gls{guw2}} & \multicolumn{1}{c|}{\gls{w1}} & \multicolumn{1}{c|}{\gls{w2}} & \multicolumn{1}{c|}{\gls{guw1}} & \multicolumn{1}{c|}{\gls{guw2}} & \multicolumn{1}{c|}{\gls{w1}} & \multicolumn{1}{c|}{\gls{w2}} \\
\hline \hline 
\hyperref[RRDD-E-5e13-A]{\texttt{\verb|RRDD-E-5e13-A|}} & \( 0.689 \) & \( 0.195 \) & \cellcolor{Mines} \textcolor{white}{\( 0.009 \)} & \( 0.048 \) & \cellcolor{Mines} \textcolor{white}{\( 0.009 \)} & \( 35680 \) & \( 28 \) & \( 33 \) & \( 9 \) \\
\hyperref[RRDD-E-5e13-B]{\texttt{\verb|RRDD-E-5e13-B|}} & \( 0.000 \) & \( 0.038 \) & \cellcolor{Mines} \textcolor{white}{\( 0.012 \)} & \( 0.035 \) & \cellcolor{Mines} \textcolor{white}{\( 0.012 \)} & \( 241 \) & \( 74 \) & \( 49 \) & \( 24 \) \\
\hyperref[RRDD-E-5e14-A]{\texttt{\verb|RRDD-E-5e14-A|}} & \( 0.194 \) & \( 0.012 \) & \cellcolor{Mines} \textcolor{white}{\( 0.011 \)} & \( 0.017 \) & \cellcolor{Mines} \textcolor{white}{\( 0.011 \)} & \( 41 \) & \( 37 \) & \( 10 \) & \( 12 \) \\
\hyperref[RRDD-E-5e14-B]{\texttt{\verb|RRDD-E-5e14-B|}} & \( 0.296 \) & \( 0.044 \) & \cellcolor{Mines} \textcolor{white}{\( 0.008 \)} & \( 0.018 \) & \cellcolor{Mines} \textcolor{white}{\( 0.008 \)} & \( 490 \) & \( 55 \) & \( 24 \) & \( 17 \) \\
\hyperref[RRDD-E-5e15-A]{\texttt{\verb|RRDD-E-5e15-A|}} & \( 0.000 \) & \( 0.002 \) & \( 0.006 \) & \cellcolor{Mines} \textcolor{white}{\( 0.001 \)} & \( 0.006 \) & \( 6 \) & \( 18 \) & \( 1 \) & \( 6 \) \\
\hyperref[RRDD-E-5e15-B]{\texttt{\verb|RRDD-E-5e15-B|}} & \( 0.000 \) & \cellcolor{Mines} \textcolor{white}{\( 0.006 \)} & \( 0.007 \) & \cellcolor{Mines} \textcolor{white}{\( 0.006 \)} & \( 0.007 \) & \( 45 \) & \( 56 \) & \( 10 \) & \( 18 \) \\
\hline
\hyperref[RRDD-R-5e13-A]{\texttt{\verb|RRDD-R-5e13-A|}} & \( 0.829 \) & \( 0.030 \) & \cellcolor{Mines} \textcolor{white}{\( 0.008 \)} & \( 0.038 \) & \cellcolor{Mines} \textcolor{white}{\( 0.008 \)} & \( 2358 \) & \( 395 \) & \( 697 \) & \( 126 \) \\
\hyperref[RRDD-R-5e13-B]{\texttt{\verb|RRDD-R-5e13-B|}} & \( 1.553 \) & \( 0.121 \) & \cellcolor{Mines} \textcolor{white}{\( 0.007 \)} & \( 0.036 \) & \cellcolor{Mines} \textcolor{white}{\( 0.007 \)} & \( 33461 \) & \( 277 \) & \( 341 \) & \( 88 \) \\
\hyperref[RRDD-R-5e14-A]{\texttt{\verb|RRDD-R-5e14-A|}} & \( 0.518 \) & \( 0.023 \) & \cellcolor{Mines} \textcolor{white}{\( 0.011 \)} & \( 0.028 \) & \cellcolor{Mines} \textcolor{white}{\( 0.011 \)} & \( 1679 \) & \( 543 \) & \( 377 \) & \( 173 \) \\
\hyperref[RRDD-R-5e14-B]{\texttt{\verb|RRDD-R-5e14-B|}} & \( 0.642 \) & \( 0.015 \) & \cellcolor{Mines} \textcolor{white}{\( 0.004 \)} & \( 0.004 \) & \cellcolor{Mines} \textcolor{white}{\( 0.004 \)} & \( 373 \) & \( 133 \) & \( 31 \) & \( 42 \) \\
\hyperref[RRDD-R-5e15-A]{\texttt{\verb|RRDD-R-5e15-A|}} & \( 0.000 \) & \cellcolor{Mines} \textcolor{white}{\( 0.006 \)} & \( 0.008 \) & \cellcolor{Mines} \textcolor{white}{\( 0.006 \)} & \( 0.008 \) & \( 307 \) & \( 445 \) & \( 68 \) & \( 142 \) \\
\hyperref[RRDD-R-5e15-B]{\texttt{\verb|RRDD-R-5e15-B|}} & \( 0.133 \) & \( 0.017 \) & \( 0.006 \) & \cellcolor{Mines} \textcolor{white}{\( 0.001 \)} & \( 0.006 \) & \( 848 \) & \( 225 \) & \( 5 \) & \( 72 \) \\
\hline
\hyperref[RDD-5e13]{\texttt{\verb|RDD-5e13|}} & \( 0.000 \) & \( 0.017 \) & \cellcolor{Mines} \textcolor{white}{\( 0.008 \)} & \( 0.017 \) & \cellcolor{Mines} \textcolor{white}{\( 0.008 \)} & \( 484 \) & \( 246 \) & \( 107 \) & \( 78 \) \\
\hyperref[RDD-5e14]{\texttt{\verb|RDD-5e14|}} & \( 1.241 \) & \( 0.088 \) & \cellcolor{Mines} \textcolor{white}{\( 0.005 \)} & \( 0.019 \) & \cellcolor{Mines} \textcolor{white}{\( 0.005 \)} & \( 17813 \) & \( 215 \) & \( 178 \) & \( 68 \) \\
\hyperref[RDD-5e15]{\texttt{\verb|RDD-5e15|}} & \( 0.000 \) & \cellcolor{Mines} \textcolor{white}{\( 0.004 \)} & \( 0.008 \) & \( 0.004 \) & \( 0.008 \) & \( 126 \) & \( 346 \) & \( 34 \) & \( 110 \) \\
\hline
\hyperref[RCDD-D-5e13-A]{\texttt{\verb|RCDD-D-5e13-A|}} & \( 0.582 \) & \( 0.219 \) & \cellcolor{Mines} \textcolor{white}{\( 0.010 \)} & \( \rightarrow \infty \) & \cellcolor{Mines} \textcolor{white}{\( 0.010 \)} & \( 143 \) & \( 5 \) & \( 10 \) & \( 2 \) \\
\hyperref[RCDD-D-5e13-B]{\texttt{\verb|RCDD-D-5e13-B|}} & \( 0.452 \) & \( 0.107 \) & \cellcolor{Mines} \textcolor{white}{\( 0.013 \)} & \( 0.559 \) & \cellcolor{Mines} \textcolor{white}{\( 0.013 \)} & \( 167 \) & \( 18 \) & \( 15 \) & \( 6 \) \\
\hyperref[RCDD-D-5e13-C]{\texttt{\verb|RCDD-D-5e13-C|}} & \( 0.175 \) & \( 0.064 \) & \cellcolor{Mines} \textcolor{white}{\( 0.008 \)} & \( 0.155 \) & \cellcolor{Mines} \textcolor{white}{\( 0.008 \)} & \( 383 \) & \( 38 \) & \( 35 \) & \( 12 \) \\
\hyperref[RCDD-D-5e14-A]{\texttt{\verb|RCDD-D-5e14-A|}} & \( 1.002 \) & \( 0.351 \) & \cellcolor{Mines} \textcolor{white}{\( 0.013 \)} & \( \rightarrow \infty \) & \cellcolor{Mines} \textcolor{white}{\( 0.013 \)} & \( 86 \) & \( 1 \) & \( 1 \) & \( 0 \) \\
\hyperref[RCDD-D-5e14-B]{\texttt{\verb|RCDD-D-5e14-B|}} & \( 1.434 \) & \( 0.366 \) & \cellcolor{Mines} \textcolor{white}{\( 0.025 \)} & \( \rightarrow \infty \) & \cellcolor{Mines} \textcolor{white}{\( 0.025 \)} & \( 88 \) & \( 3 \) & \( 2 \) & \( 1 \) \\
\hyperref[RCDD-D-5e14-C]{\texttt{\verb|RCDD-D-5e14-C|}} & \( 1.499 \) & \( 0.333 \) & \cellcolor{Mines} \textcolor{white}{\( 0.032 \)} & \( \rightarrow \infty \) & \cellcolor{Mines} \textcolor{white}{\( 0.032 \)} & \( 199 \) & \( 9 \) & \( 4 \) & \( 3 \) \\
\hyperref[RCDD-D-5e15-A]{\texttt{\verb|RCDD-D-5e15-A|}} & \( 0.476 \) & \( 0.381 \) & \cellcolor{Mines} \textcolor{white}{\( 0.011 \)} & \( \rightarrow \infty \) & \cellcolor{Mines} \textcolor{white}{\( 0.011 \)} & \( 46 \) & \( 0 \) & \( 0 \) & \( 0 \) \\
\hyperref[RCDD-D-5e15-B]{\texttt{\verb|RCDD-D-5e15-B|}} & \( 1.012 \) & \( 0.433 \) & \cellcolor{Mines} \textcolor{white}{\( 0.006 \)} & \( \rightarrow \infty \) & \cellcolor{Mines} \textcolor{white}{\( 0.006 \)} & \( 46 \) & \( 0 \) & \( 0 \) & \( 0 \) \\
\hyperref[RCDD-D-5e15-C]{\texttt{\verb|RCDD-D-5e15-C|}} & \( 0.507 \) & \( 0.515 \) & \cellcolor{Mines} \textcolor{white}{\( 0.001 \)} & \( \rightarrow \infty \) & \cellcolor{Mines} \textcolor{white}{\( 0.001 \)} & \( 47 \) & \( 0 \) & \( 0 \) & \( 0 \) \\
\hline
\hyperref[RCDD-E-5e13]{\texttt{\verb|RCDD-E-5e13|}} & \( 1.149 \) & \( 0.097 \) & \cellcolor{Mines} \textcolor{white}{\( 0.012 \)} & \( 0.102 \) & \cellcolor{Mines} \textcolor{white}{\( 0.012 \)} & \( 533 \) & \( 27 \) & \( 28 \) & \( 9 \) \\
\hyperref[RCDD-E-5e14]{\texttt{\verb|RCDD-E-5e14|}} & \( 3.841 \) & \( 0.282 \) & \cellcolor{Mines} \textcolor{white}{\( 0.037 \)} & \( 0.183 \) & \cellcolor{Mines} \textcolor{white}{\( 0.037 \)} & \( 2143 \) & \( 18 \) & \( 6 \) & \( 6 \) \\
\hyperref[RCDD-E-5e15]{\texttt{\verb|RCDD-E-5e15|}} & \( 0.588 \) & \( 0.284 \) & \cellcolor{Mines} \textcolor{white}{\( 0.008 \)} & \( 0.392 \) & \cellcolor{Mines} \textcolor{white}{\( 0.008 \)} & \( 21 \) & \( 0 \) & \( 2 \) & \( 0 \) \\
\hline
\hyperref[RCDD-R-5e13]{\texttt{\verb|RCDD-R-5e13|}} & \( 0.008 \) & \( 0.072 \) & \cellcolor{Mines} \textcolor{white}{\( 0.014 \)} & \( 0.079 \) & \cellcolor{Mines} \textcolor{white}{\( 0.014 \)} & \( 166 \) & \( 56 \) & \( 35 \) & \( 18 \) \\
\hyperref[RCDD-R-5e14]{\texttt{\verb|RCDD-R-5e14|}} & \( 4.186 \) & \( 0.310 \) & \cellcolor{Mines} \textcolor{white}{\( 0.036 \)} & \( 0.150 \) & \cellcolor{Mines} \textcolor{white}{\( 0.036 \)} & \( 5047 \) & \( 31 \) & \( 9 \) & \( 10 \) \\
\hyperref[RCDD-R-5e15]{\texttt{\verb|RCDD-R-5e15|}} & \( 0.653 \) & \( 0.180 \) & \cellcolor{Mines} \textcolor{white}{\( 0.006 \)} & \( 0.256 \) & \cellcolor{Mines} \textcolor{white}{\( 0.006 \)} & \( 24 \) & \( 0 \) & \( 2 \) & \( 0 \) \\
\hline
\end{tabularx}
\end{center}

\captionof{table}{Effective cut-off radius sandard deviation and relative density standard deviation for harmonic \( j = 1 \) of \gls{guw1}, \gls{guw2}, \gls{w1} and \gls{w2} models fitted on the X-ray simulation output performed on distributions generated in a square \gls{roi} of 3200 nm side length with \gls{pbc}1.}

\newpage

\begin{center}
\begin{tabularx}{\linewidth}{|l|l|>{\raggedleft\arraybackslash}X|>{\raggedleft\arraybackslash}X|>{\raggedleft\arraybackslash}X|>{\raggedleft\arraybackslash}X|>{\raggedleft\arraybackslash}X|>{\raggedleft\arraybackslash}X|>{\raggedleft\arraybackslash}X|>{\raggedleft\arraybackslash}X|} 
\hline
\multirow{2}{*}{\centering{Distribution}} & \multicolumn{1}{c|}{\centering{$ \hat{\gls{stddev}} \left( \delta \right) $}} & \multicolumn{4}{c|}{ $ \left. \hat{\gls{stddev}} \left( \gls{dst}^{\mathrm{FIT}} \right) \right/ \gls{dst} $} & \multicolumn{4}{c|}{$ \hat{\gls{stddev}} \left( \gls{cutrad}^{\mathrm{FIT}} \right) $ (nm)} \\
\cline{2-10}
 & \multicolumn{2}{c|}{\gls{guw1}} & \multicolumn{1}{c|}{\gls{guw2}} & \multicolumn{1}{c|}{\gls{w1}} & \multicolumn{1}{c|}{\gls{w2}} & \multicolumn{1}{c|}{\gls{guw1}} & \multicolumn{1}{c|}{\gls{guw2}} & \multicolumn{1}{c|}{\gls{w1}} & \multicolumn{1}{c|}{\gls{w2}} \\
\hline \hline 
\hyperref[RRDD-E-5e13-A]{\texttt{\verb|RRDD-E-5e13-A|}} & \( 7.383 \) & \( 0.090 \) & \cellcolor{Mines} \textcolor{white}{\( 0.007 \)} & \( 0.017 \) & \cellcolor{Mines} \textcolor{white}{\( 0.007 \)} & \( 18130 \) & \( 5 \) & \( 6 \) & \( 3 \) \\
\hyperref[RRDD-E-5e13-B]{\texttt{\verb|RRDD-E-5e13-B|}} & \( 0.000 \) & \( 0.025 \) & \cellcolor{Mines} \textcolor{white}{\( 0.009 \)} & \( 0.023 \) & \cellcolor{Mines} \textcolor{white}{\( 0.009 \)} & \( 70 \) & \( 12 \) & \( 14 \) & \( 8 \) \\
\hyperref[RRDD-E-5e14-A]{\texttt{\verb|RRDD-E-5e14-A|}} & \( 0.096 \) & \cellcolor{Mines} \textcolor{white}{\( 0.007 \)} & \( 0.009 \) & \( 0.010 \) & \( 0.009 \) & \( 9 \) & \( 6 \) & \( 3 \) & \( 4 \) \\
\hyperref[RRDD-E-5e14-B]{\texttt{\verb|RRDD-E-5e14-B|}} & \( 0.407 \) & \( 0.028 \) & \cellcolor{Mines} \textcolor{white}{\( 0.010 \)} & \( 0.012 \) & \cellcolor{Mines} \textcolor{white}{\( 0.010 \)} & \( 153 \) & \( 15 \) & \( 7 \) & \( 9 \) \\
\hyperref[RRDD-E-5e15-A]{\texttt{\verb|RRDD-E-5e15-A|}} & \( 0.000 \) & \( 0.002 \) & \( 0.011 \) & \cellcolor{Mines} \textcolor{white}{\( 0.000 \)} & \( 0.011 \) & \( 3 \) & \( 6 \) & \( 0 \) & \( 4 \) \\
\hyperref[RRDD-E-5e15-B]{\texttt{\verb|RRDD-E-5e15-B|}} & \( 0.000 \) & \cellcolor{Mines} \textcolor{white}{\( 0.002 \)} & \( 0.009 \) & \cellcolor{Mines} \textcolor{white}{\( 0.002 \)} & \( 0.009 \) & \( 7 \) & \( 17 \) & \( 1 \) & \( 11 \) \\
\hline
\hyperref[RRDD-R-5e13-A]{\texttt{\verb|RRDD-R-5e13-A|}} & \( 0.664 \) & \( 0.027 \) & \cellcolor{Mines} \textcolor{white}{\( 0.007 \)} & \( 0.017 \) & \cellcolor{Mines} \textcolor{white}{\( 0.007 \)} & \( 1520 \) & \( 95 \) & \( 135 \) & \( 61 \) \\
\hyperref[RRDD-R-5e13-B]{\texttt{\verb|RRDD-R-5e13-B|}} & \( 3.478 \) & \( 0.105 \) & \cellcolor{Mines} \textcolor{white}{\( 0.008 \)} & \( 0.017 \) & \cellcolor{Mines} \textcolor{white}{\( 0.008 \)} & \( 31927 \) & \( 61 \) & \( 61 \) & \( 39 \) \\
\hyperref[RRDD-R-5e14-A]{\texttt{\verb|RRDD-R-5e14-A|}} & \( 0.138 \) & \( 0.012 \) & \cellcolor{Mines} \textcolor{white}{\( 0.006 \)} & \( 0.013 \) & \cellcolor{Mines} \textcolor{white}{\( 0.006 \)} & \( 455 \) & \( 74 \) & \( 86 \) & \( 47 \) \\
\hyperref[RRDD-R-5e14-B]{\texttt{\verb|RRDD-R-5e14-B|}} & \( 0.355 \) & \( 0.024 \) & \( 0.008 \) & \cellcolor{Mines} \textcolor{white}{\( 0.003 \)} & \( 0.008 \) & \( 529 \) & \( 61 \) & \( 12 \) & \( 39 \) \\
\hyperref[RRDD-R-5e15-A]{\texttt{\verb|RRDD-R-5e15-A|}} & \( 0.000 \) & \cellcolor{Mines} \textcolor{white}{\( 0.000 \)} & \( 0.007 \) & \cellcolor{Mines} \textcolor{white}{\( 0.000 \)} & \( 0.007 \) & \( 4 \) & \( 102 \) & \( 1 \) & \( 65 \) \\
\hyperref[RRDD-R-5e15-B]{\texttt{\verb|RRDD-R-5e15-B|}} & \( 1.148 \) & \( 0.114 \) & \( 0.018 \) & \cellcolor{Mines} \textcolor{white}{\( 0.000 \)} & \( 0.018 \) & \( 47303 \) & \( 138 \) & \( 1 \) & \( 88 \) \\
\hline
\hyperref[RDD-5e13]{\texttt{\verb|RDD-5e13|}} & \( 0.000 \) & \( 0.010 \) & \( 0.011 \) & \cellcolor{Mines} \textcolor{white}{\( 0.010 \)} & \( 0.011 \) & \( 124 \) & \( 74 \) & \( 27 \) & \( 47 \) \\
\hyperref[RDD-5e14]{\texttt{\verb|RDD-5e14|}} & \( 1.771 \) & \( 0.089 \) & \cellcolor{Mines} \textcolor{white}{\( 0.007 \)} & \( 0.016 \) & \cellcolor{Mines} \textcolor{white}{\( 0.007 \)} & \( 13022 \) & \( 66 \) & \( 71 \) & \( 42 \) \\
\hyperref[RDD-5e15]{\texttt{\verb|RDD-5e15|}} & \( 0.000 \) & \cellcolor{Mines} \textcolor{white}{\( 0.000 \)} & \( 0.023 \) & \cellcolor{Mines} \textcolor{white}{\( 0.000 \)} & \( 0.023 \) & \( 8 \) & \( 244 \) & \( 2 \) & \( 156 \) \\
\hline
\hyperref[RCDD-D-5e13-A]{\texttt{\verb|RCDD-D-5e13-A|}} & \( 0.334 \) & \( 0.275 \) & \cellcolor{Mines} \textcolor{white}{\( 0.009 \)} & \( \rightarrow \infty \) & \cellcolor{Mines} \textcolor{white}{\( 0.009 \)} & \( 186 \) & \( 1 \) & \( 3 \) & \( 0 \) \\
\hyperref[RCDD-D-5e13-B]{\texttt{\verb|RCDD-D-5e13-B|}} & \( 0.238 \) & \( 0.197 \) & \cellcolor{Mines} \textcolor{white}{\( 0.007 \)} & \( 0.254 \) & \cellcolor{Mines} \textcolor{white}{\( 0.007 \)} & \( 229 \) & \( 1 \) & \( 3 \) & \( 1 \) \\
\hyperref[RCDD-D-5e13-C]{\texttt{\verb|RCDD-D-5e13-C|}} & \( 0.173 \) & \( 0.084 \) & \cellcolor{Mines} \textcolor{white}{\( 0.012 \)} & \( 0.069 \) & \cellcolor{Mines} \textcolor{white}{\( 0.012 \)} & \( 180 \) & \( 7 \) & \( 7 \) & \( 4 \) \\
\hyperref[RCDD-D-5e14-A]{\texttt{\verb|RCDD-D-5e14-A|}} & \( 7.844 \) & \( 0.307 \) & \cellcolor{Mines} \textcolor{white}{\( 0.019 \)} & \( \rightarrow \infty \) & \cellcolor{Mines} \textcolor{white}{\( 0.019 \)} & \( 1236 \) & \( 0 \) & \( 0 \) & \( 0 \) \\
\hyperref[RCDD-D-5e14-B]{\texttt{\verb|RCDD-D-5e14-B|}} & \( 0.330 \) & \( 0.383 \) & \cellcolor{Mines} \textcolor{white}{\( 0.009 \)} & \( \rightarrow \infty \) & \cellcolor{Mines} \textcolor{white}{\( 0.009 \)} & \( 100 \) & \( 0 \) & \( 0 \) & \( 0 \) \\
\hyperref[RCDD-D-5e14-C]{\texttt{\verb|RCDD-D-5e14-C|}} & \( 0.861 \) & \( 0.353 \) & \cellcolor{Mines} \textcolor{white}{\( 0.012 \)} & \( \rightarrow \infty \) & \cellcolor{Mines} \textcolor{white}{\( 0.012 \)} & \( 88 \) & \( 0 \) & \( 1 \) & \( 0 \) \\
\hyperref[RCDD-D-5e15-A]{\texttt{\verb|RCDD-D-5e15-A|}} & \( 0.149 \) & \( 0.162 \) & \( 0.052 \) & \cellcolor{Mines} \textcolor{white}{\( 0.002 \)} & \( 0.052 \) & \( 54 \) & \( 0 \) & \( 0 \) & \( 0 \) \\
\hyperref[RCDD-D-5e15-B]{\texttt{\verb|RCDD-D-5e15-B|}} & \( 0.374 \) & \( 0.319 \) & \( 0.068 \) & \cellcolor{Mines} \textcolor{white}{\( 0.006 \)} & \( 0.068 \) & \( 49 \) & \( 0 \) & \( 0 \) & \( 0 \) \\
\hyperref[RCDD-D-5e15-C]{\texttt{\verb|RCDD-D-5e15-C|}} & \( 0.324 \) & \( 0.389 \) & \( 0.080 \) & \cellcolor{Mines} \textcolor{white}{\( 0.007 \)} & \( 0.080 \) & \( 53 \) & \( 0 \) & \( 0 \) & \( 0 \) \\
\hline
\hyperref[RCDD-E-5e13]{\texttt{\verb|RCDD-E-5e13|}} & \( 0.486 \) & \( 0.063 \) & \cellcolor{Mines} \textcolor{white}{\( 0.012 \)} & \( 0.096 \) & \cellcolor{Mines} \textcolor{white}{\( 0.012 \)} & \( 137 \) & \( 4 \) & \( 16 \) & \( 3 \) \\
\hyperref[RCDD-E-5e14]{\texttt{\verb|RCDD-E-5e14|}} & \( 0.499 \) & \( 0.093 \) & \cellcolor{Mines} \textcolor{white}{\( 0.008 \)} & \( 0.175 \) & \cellcolor{Mines} \textcolor{white}{\( 0.008 \)} & \( 22 \) & \( 0 \) & \( 3 \) & \( 0 \) \\
\hyperref[RCDD-E-5e15]{\texttt{\verb|RCDD-E-5e15|}} & \( 0.118 \) & \( 0.269 \) & \cellcolor{Mines} \textcolor{white}{\( 0.087 \)} & \( 0.233 \) & \cellcolor{Mines} \textcolor{white}{\( 0.087 \)} & \( 8 \) & \( 1 \) & \( 1 \) & \( 0 \) \\
\hline
\hyperref[RCDD-R-5e13]{\texttt{\verb|RCDD-R-5e13|}} & \( 0.572 \) & \( 0.056 \) & \cellcolor{Mines} \textcolor{white}{\( 0.010 \)} & \( 0.072 \) & \cellcolor{Mines} \textcolor{white}{\( 0.010 \)} & \( 130 \) & \( 6 \) & \( 18 \) & \( 4 \) \\
\hyperref[RCDD-R-5e14]{\texttt{\verb|RCDD-R-5e14|}} & \( 1.177 \) & \( 0.174 \) & \cellcolor{Mines} \textcolor{white}{\( 0.009 \)} & \( 0.138 \) & \cellcolor{Mines} \textcolor{white}{\( 0.009 \)} & \( 231 \) & \( 1 \) & \( 5 \) & \( 0 \) \\
\hyperref[RCDD-R-5e15]{\texttt{\verb|RCDD-R-5e15|}} & \( 0.267 \) & \( 0.158 \) & \cellcolor{Mines} \textcolor{white}{\( 0.078 \)} & \( 0.136 \) & \cellcolor{Mines} \textcolor{white}{\( 0.078 \)} & \( 7 \) & \( 1 \) & \( 0 \) & \( 1 \) \\
\hline
\end{tabularx}
\end{center}

\captionof{table}{Effective cut-off radius sandard deviation and relative density standard deviation for harmonic \( j = 2 \) of \gls{guw1}, \gls{guw2}, \gls{w1} and \gls{w2} models fitted on the X-ray simulation output performed on distributions generated in a square \gls{roi} of 3200 nm side length with \gls{pbc}1.}

\newpage



\medfig{fig:std-RRDD-E-A}{load/plots}{std-RRDD-E-A}{RRDD-E-A}%
\medfig{fig:std-RRDD-E-B}{load/plots}{std-RRDD-E-B}{RRDD-E-B}%

\medfig{fig:std-RRDD-R-A}{load/plots}{std-RRDD-R-A}{RRDD-R-A}%
\medfig{fig:std-RRDD-R-B}{load/plots}{std-RRDD-R-B}{RRDD-R-B}%

\medfig{fig:std-RDD}{load/plots}{std-RDD}{RDD}%
\medfig{fig:std-RCDD-D-A}{load/plots}{std-RCDD-D-A}{RCDD-D-A}%

\medfig{fig:std-RCDD-D-B}{load/plots}{std-RCDD-D-B}{RCDD-D-B}%
\medfig{fig:std-RCDD-D-C}{load/plots}{std-RCDD-D-C}{RCDD-D-C}%

\medfig{fig:std-RCDD-E}{load/plots}{std-RCDD-E}{RCDD-E}%
\medfig{fig:std-RCDD-R}{load/plots}{std-RCDD-R}{RCDD-R}%



\newpage

\subsubsection{Validity domain of \glsentrytext{guw1}}

\medfig{guw1-RRDD-E-A}{load/plots}{guw1-RRDD-E-A}{RRDD-E-A}%
\medfig{guw1-RRDD-E-B}{load/plots}{guw1-RRDD-E-B}{RRDD-E-B}%

\medfig{guw1-RRDD-R-A}{load/plots}{guw1-RRDD-R-A}{RRDD-R-A}%
\medfig{guw1-RRDD-R-B}{load/plots}{guw1-RRDD-R-B}{RRDD-R-B}%

\medfig{guw1-RDD}{load/plots}{guw1-RDD}{RDD}%
\medfig{guw1-RCDD-D-A}{load/plots}{guw1-RCDD-D-A}{RCDD-D-A}%

\medfig{guw1-RCDD-D-B}{load/plots}{guw1-RCDD-D-B}{RCDD-D-B}%
\medfig{guw1-RCDD-D-C}{load/plots}{guw1-RCDD-D-C}{RCDD-D-C}%

\medfig{guw1-RCDD-E}{load/plots}{guw1-RCDD-E}{RCDD-E}%
\medfig{guw1-RCDD-R}{load/plots}{guw1-RCDD-R}{RCDD-R}%



\newpage

\subsection{Program}

The third and last program is used to process the post-simulation data.
It allows to average the coefficients from the Fourier analysis, to plot the simulation results, and to export the data and figures of the fits performed for each model.

\subsubsection{Repository}

Useful information can be found on the project page: \github{lpa-output}

\subsubsection{Installation}

The program can be installed or updated with the following command:

\pipinstall{lpa-output}

\subsubsection{Fits data}

For each harmonic and each maximum value of \gls{frrvar}, the optimal density and outer cut-off radius in terms of fit are saved in a file like the one below. These files are produced for each fit model applied to each distribution model.

\txtlst{insert/fits}{fits\_data\_structure.dat}
