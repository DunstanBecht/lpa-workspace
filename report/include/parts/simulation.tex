\section{Simulation}

\subsection{Principle}

This section introduces the main aspects of the X-ray diffraction theory and how it is numerically simulated here.

\subsubsection{The diffracted intensity profile and its Fourier transform}

Let \gls{difint} be the \glsdesc{difint} as defined by \textcite{W1979}.
\gls{difint} is a function of \gls{radcdn} the \glsdesc{radcdn} defined in \eqref{eq:S}.
\gls{wavlen} is the \glsdesc{wavlen}, \( \gls{brgang}_B \) is the Bragg angle of the average lattice defined in \eqref{eq:Bragg-angle} and \gls{celprm} is the \glsdesc{celprm}.
\gls{difint} is normalized according to \eqref{eq:norm-I}.

\begin{equation}\label{eq:Bragg-angle}
\sin \gls{brgang}_B = \frac{\gls{wavlen} \sqrt{h^2+k^2+l^2}}{2 \gls{celprm}}
\end{equation}

\begin{equation}\label{eq:S}
\gls{radcdn} = \frac{2 \left( \sin \gls{brgang} - \sin \gls{brgang}_B \right)}{\gls{wavlen}}
\end{equation}

\begin{equation}\label{eq:norm-I}
\int \gls{difint}(\gls{radcdn}) d\gls{radcdn} = 1
\end{equation}

The \glsdesc{difint} \gls{difint} can be represented in terms of its Fourier transform coefficients \gls{frrcfa} and \gls{frrcfb} for \( {\gls{hmc}} \in \gls{setN}^*_+ \) according to \eqref{eq:fourier-a} and \eqref{eq:fourier-b}.

\begin{equation}\label{eq:fourier-a}
\gls{frrcfa}(\gls{frrvar}) = \int \gls{difint}(\gls{radcdn}) \cos \left( -2 \pi \gls{hmc} \gls{radcdn} \gls{frrvar} \right) d\gls{radcdn}
\end{equation}

\begin{equation}\label{eq:fourier-b}
\gls{frrcfb}(\gls{frrvar}) = \int \gls{difint}(\gls{radcdn}) \sin \left( -2 \pi \gls{hmc} \gls{radcdn} \gls{frrvar} \right) d\gls{radcdn}
\end{equation}

Let \gls{vecu} be the \glsdesc{vecu}.
According to the kinematical theory of X-ray scattering \cite{W1990}:

\begin{equation}\label{eq:fourier-a-warren}
\gls{frrcfa}(\gls{frrvar}) = \frac{\int_{\gls{roi}} \cos \left( 2 \pi \gls{hmc} \gls{vecg} \cdot \left[ \gls{vecu}(\vec{r} + \gls{frrvar} \overrightarrow{u_x}) - \gls{vecu}(\vec{r}) \right] \right) d\vec{r}}{\gls{area}_{\gls{roi}}}
\end{equation}

\begin{equation}\label{eq:fourier-b-warren}
\gls{frrcfb}(\gls{frrvar}) = \frac{\int_{\gls{roi}} \sin \left( 2 \pi \gls{hmc} \gls{vecg} \cdot \left[ \gls{vecu}(\vec{r} + \gls{frrvar} \overrightarrow{u_x}) - \gls{vecu}(\vec{r}) \right] \right) d\vec{r}}{\gls{area}_{\gls{roi}}}
\end{equation}

\medskip

The resultant displacement field created by a dislocation located in \( ( x, y ) \) with respect to the observation position is given in equation \eqref{eq:displacement-field}.

\begin{align}\label{eq:displacement-field}
\gls{vecu} =
  \left( \begin{array}{l}
    \gls{indfun}_{\mathrm{edge}} \frac{| \gls{vecb} |}{2 \pi} \left( \arctan \left( \frac{y}{x} \right) + \frac{x y}{2 (1 - \gls{pssrat}) \left( x^2 + y^2 \right)} \right)
    \\[5mm]
    - \gls{indfun}_{\mathrm{edge}} \frac{| \gls{vecb} |}{8 \pi (1 - \gls{pssrat})} \left( (1 - 2 \gls{pssrat}) \ln \left( x^2 + y^2 \right) + \frac{x^2 - y^2}{x^2 + y^2} \right)
    \\[5mm]
    \gls{indfun}_{\mathrm{screw}} \frac{| \gls{vecb} |}{2 \pi} \arctan \left( \frac{y}{x} \right)
  \end{array} \right)
\end{align}

\subsubsection{Parallelization}

To calculate \( \gls{frrcfa}(\gls{frrvar}) \) and \( \gls{frrcfb}(\gls{frrvar}) \), the Monte Carlo method is used.
A large number of random points \( \vec{r} \) are drawn in the crystal and the equations \eqref{eq:fourier-a-warren} and \eqref{eq:fourier-b-warren} are applied. The random draws and the calculations are then parallelized since they are independent.

\subsection{Parameters choice}

Many parameters related to the Monte-Carlo approach and simulation choices have to be defined.
This section aims to present the methods used to determine their value.

\subsubsection{Periodic boundary condition}

To determine the number of replications of the \gls{roi} we simulate the same distribution but with different number of replications.
The optimal number is the smallest one that will still allow to obtain a Fourier transform close enough to the converged value.
\figref{fig:pbc-converge-1} and \figref{fig:pbc-converge-2} show that the optimal number of replications for a \gls{roi} of 6400 nm is 1.

\medfig{fig:pbc-converge-1}{\datdir/superposition_RRDD-E_test_PBC}{RRDD-E_d_1.25e13_m-2}{Convergence of \gls{frrtrf} as a function of \gls{pbc}}%
\medfig{fig:pbc-converge-2}{\datdir/superposition_RRDD-E_test_PBC}{RRDD-E_d_1.28e16_m-2}{Convergence of \gls{frrtrf} as a function of \gls{pbc}}%

\subsubsection{Number of input files and random points}

Let \( \vec{R} \) be a random variable corresponding to a random position in the \gls{roi}.
Let \( X \) be a random variable corresponding to the mesure of a Fourier coefficient (say \( \cos \)) at a given \gls{hmc}, \gls{vecg} and \gls{frrvar}.
Let \( Y \) be the average of \( N_P \) values of \( X \).
The value found in an output file of the simulator is a random draw of \( Y \).
Let \( Z \) be the average of \( N_F \) values of \( Y \).
The value deduced from \( N_F \) output files is a random draw of \( Z \).
Let \( \gls{stddev}_X \), \( \gls{stddev}_Y \), \( \gls{stddev}_Z \), be the standard deviations of \( X \), \( Y \), \( Z \).

\begin{equation}
X
= \cos \left( 2 \pi \gls{hmc} \gls{vecg} \cdot \left[ \gls{vecu}( \vec{R} + L \overrightarrow{u_x}) - \gls{vecu}(\vec{R}) \right] \right)
\end{equation}

\smallskip

\begin{minipage}{0.5\linewidth}

\begin{equation}
Y
= \frac{1}{N_P} \sum_{i=1}^{N_P} X_i
\end{equation}

\smallskip

\begin{equation}
Z
= \frac{1}{N_F} \sum_{i=1}^{N_F} Y_i
\end{equation}

\end{minipage}%
\begin{minipage}{0.5\linewidth}

\begin{equation}
\gls{stddev}_Y
= \frac{\gls{stddev}_X}{\sqrt{N_P}}
\end{equation}

\smallskip

\begin{equation}\label{eq:fourier-error}
\gls{stddev}_Z
= \frac{\gls{stddev}_Y}{\sqrt{N_F}}
= \frac{\gls{stddev}_X}{\sqrt{N_F N_P}}
\end{equation}

\end{minipage}

\bigskip

The standard error of the Fourier transform, at the end of the process, after all the averaging carried out, is equal to \( \gls{stddev}_Z \).
In practice it can be observed that for a given \( N_P N_F \), \( \gls{frrvar} \sqrt{\gls{dst}} \), \( s \sqrt{\gls{dst}} \), and \gls{roi} size, \( \gls{stddev}_Z \) gives the same value regardless of the density.
That is an important result for the following.
It means that to maintain consistency between the results obtained for different densities, \( N_P N_F \) should be constant.

\medskip

To evaluate the optimal value of \( N_P N_F \), we will look at the asymptotic behavior of the Fourier transform.
The Fourier transform should have a linear behavior in representation \acrshort{r2} (see \ref{sec:representation-2}).
The impact of \( N_P N_F \) and the number of availables points for the fit should be dissociated for a good appreciation of the precision.

\medskip

Since the interval with linear behavior seems to be proportional to \( 1 / \sqrt{\gls{dst}} \) (see \ref{sec:filters}) we will take a step \gls{frrstp} such as \( \gls{frrstp} \sqrt{\gls{dst}} \) is constant in order to obtain an equivalent number of values of the Fourier transform in this interval.
This makes the perception and appreciation of the asymptotic behavior equal for each density.

\subsubsection{Step}

The step for \gls{frrvar} is called \gls{frrstp} and is given in equation \eqref{eq:fourier-a3} \cite{W1990}.
There are experimental constraints on \gls{brgang} that limit the range of available values for \gls{frrstp}.
The minimum value of \gls{frrstp} is determined by the maximum value of \( | \gls{brgang} - \gls{brgang}_B | = \Delta \gls{brgang} / 2 \).
And the maximum value of \gls{frrstp} is determined by the minimum value of \( \Delta \gls{brgang} / 2 \).
The minimum value of \( \Delta \gls{brgang} / 2 \) is arbitrary.
The maximum value of \( \Delta \gls{brgang} / 2 \) is related to the proximity of neighboring peaks (centered on the contiguous Bragg angles).
Several peaks should not overlap in the angular area under study.
We will fix here the upper limit of \( \Delta \gls{brgang} / 2 \) at half the angular distance of the closest Bragg angle.

\begin{equation}\label{eq:fourier-L-step}
\gls{frrvar} = i \gls{frrstp} \ \text{with} \ i \in \gls{setN}^*_+
\end{equation}

\begin{equation}\label{eq:fourier-a3}
\gls{frrstp} = \frac{\gls{wavlen}}{4 \left( \sin \gls{brgang} - \sin \gls{brgang}_B \right)}
\end{equation}

\medskip

Assuming \( \gls{wavlen} = 0.154 \) nm (Cu) and \( \gls{celprm} =  0.4046 \) nm  (for Al FCC), we get the following table.

\medskip

{\renewcommand{\arraystretch}{1.6}

\begin{tabularx}{\linewidth}{|X|X|X|X|X|X|}
\hline
hkl & \( \gls{brgang}_B \) & \( \max \left( \Delta \theta / 2 \right) \) \(^{\circ}\) & \( \min \left( a_3 \right) \) nm & \( \min \left( \Delta \theta / 2 \right) \) \(^{\circ}\) & \( \max \left( a_3 \right) \) nm \\
\hline
111 & 19.25 & 1.56 & 1.50 & 0.50 & 4.68 \\
\hline
200 & 22.37 & 1.56 & 1.53 & 0.50 & 4.78 \\
\hline
220 & 32.57 & 3.29 & 0.81 & 0.50 & 5.25 \\
\hline
311 & 39.14 & 1.05 & 2.72 & 0.50 & 5.71 \\
\hline
222 & 41.24 & 1.05 & 2.81 & 0.50 & 5.89 \\
\hline
400 & 49.57 & 3.24 & 1.09 & 0.50 & 6.84 \\
\hline
331 & 56.05 & 1.14 & 3.52 & 0.50 & 7.95 \\
\hline
420 & 58.33 & 1.14 & 3.75 & 0.50 & 8.46 \\
\hline
422 & 68.80 & 5.23 & 1.32 & 0.50 & 12.34 \\
\hline
\end{tabularx}

\captionof{table}{Range of possible values for \gls{frrstp} as a function of (hkl).}

\bigskip

Here we will use hkl (200).
The minimum realistic value of \gls{frrstp} will be \( 1.5 \) nm and its maximum value is about \( 5 \) nm.
We can think of several ways to determine \gls{frrstp} as a function of the density:
\begin{enumerate}
\item Set \gls{frrstp} to its minimum realistic value (\( 1.5 \) nm) regardless of the density.
\begin{itemize}
\item It will give accurate results for small densities and less accurate for large ones as a consequence of the difference in the number of points available for the fit (see \ref{sec:filters}).
\item This will be very expensive in computation time for small densities.
\end{itemize}
\item Set \( \gls{frrstp} \sqrt{\gls{dst}} \) to a constant value.
\begin{itemize}
\item This does not correspond to the real experimental conditions: with \( \gls{frrstp} \sqrt{\gls{dst}} = 0.17 \) we get \( \gls{frrstp} = 1.5 \) nm for \( \gls{dst} = 1.28 \times 10^{16} \) m\( ^{-2} \) and \( \gls{frrstp} = 48.1 \) nm for \( \gls{dst} = 1.25 \times 10^{13} \) m\( ^{-2} \).
\item This approach makes it possible to establish an important result: we obtain exactly the same relative error in the density prediction at a given \( s \sqrt{\gls{dst}} \) (regardless of the density).
\item It is relatively inexpensive in computation time.
\end{itemize}
\item Set \( \gls{frrstp} = \min \left( \max \left(\gls{frrvar}_{\mathrm{max}}/n, {a_3}_{\mathrm{min}} \right), {a_3}_{\mathrm{max}} \right) \) with \( {a_3}_{\mathrm{min}} = 1.5 \) nm and \( {a_3}_{\mathrm{max}} = 5.0 \) nm.
\begin{itemize}
\item The behavior of the error as a function of the density will strongly depend on the choice of the parameters \( n \), \( {a_3}_{\mathrm{min}} \) and \( {a_3}_{\mathrm{max}} \). When looking at the results it must be kept in mind that what we see is very influenced by a discontinuity in the way of choosing \gls{frrstp} when \( \gls{frrvar}_{\mathrm{max}}/n \) goes bellow \( 1.5 \) nm or above \( 5.0 \) nm.
\item It is rather expensive in computation time.
\end{itemize}
\end{enumerate}

\subsection{Program}

The second program allows the simulation of X-ray diffraction and Fourier analysis.
It was implemented with OpenCL in order to take advantage of the high capacity of parallelization of the calculations on a graphics card.

\subsubsection{Repository}

Useful information can be found on the project page: \github{lpa-xrd}

\subsubsection{Installation}

The program can be installed or updated with the following command:

\pipinstall{lpa-xrd}

\subsubsection{Output data}

For each simulation a file containing the coefficients \( \gls{frrcfa}(\gls{frrvar}) \) and \( \gls{frrcfb}(\gls{frrvar}) \) for each harmonic \gls{hmc} and Fourier variable \gls{frrvar} is produced.
The structure of these files is illustrated by the example file below.

\bigskip

\pyscript{insert/output}{output\_data.py}

\medskip

\txtlst{insert/output/croped}{output\_data.dat}
