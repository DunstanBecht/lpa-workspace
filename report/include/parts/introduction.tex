\section{Introduction}

All real crystals contain imperfections which disturb locally the regular arrangement of the atoms.
The properties of crystalline solids can be significantly modified by their presence.
We are interested here in the means of evaluating the number of line defects, also called dislocations.

\medskip

Various techniques are used to study the arrangement and density of dislocations.
One of them is the X-ray diffraction, in which local differences in the scattering of X-rays are used to show up the dislocation density.
As in electron diffraction, any local bending of the lattice associated with a dislocation results in a
change in the reflection conditions and the X-rays are scattered differently in this region.

\medskip

The difference in the intensity of the diffracted X-rays can be recorded photographically.
There are then several ways to deduce the density of dislocations from the observations.
These are three of these different models whose precision we are trying to assess and compare on different types of dislocation distributions.

\medskip

These models have application hypotheses, but some are applied beyond the latter because they are considered sufficiently versatile.
The goal here is to produce a panorama of the results of these different models on a variety of types of distributions to judge the relevance of the application of such or such model on such or such distribution.

\medskip

The study does not include an experimental part and is based instead on pure numerical simulation.
To obtain the desired data, it will be necessary to build the distribution models, analyze them, simulate the X-ray diffraction, use the Fourier analysis, adjust the models to the outputs, and synthesize the results in a form allowing the appreciation of the accuracy of the different models.
