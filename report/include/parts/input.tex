\section{Input}

\subsection{Distribution models}

The input data for the X-ray diffraction line profile simulation program are different dislocation distributions.
The latter are generated according to a model and the geometry of a \gls{roi}.
The sections \ref{sec:distribution-model-rdd}, \ref{sec:distribution-model-rrdd} and \ref{sec:distribution-model-rcdd} present the models proposed in the study and their possible variants.

\subsubsection{\glsentrytext{rdd}}\label{sec:distribution-model-rdd}

In a \gls{rdd}, the dislocation positions and Burgers vector senses are randomly chosen in the \gls{roi}.
Random positioning of points is described by equations \eqref{eq:circle-random-position-theta} and \eqref{eq:circle-random-position-r} in section \ref{sec:circle-random-position} for the circle geometry.
In a square geometry, the coordinates simply follow a continuous uniform distribution law.
There are always equal numbers of positive and negative Burgers vectors.
The model takes as parameter the variable \( d \) giving the density of dislocations.
\figref{fig:rdd-example-1} and \figref{fig:rdd-example-2} show the dislocations contained in a \gls{rdd} in a circle and a square \gls{roi} geometry.

\medfig{fig:rdd-example-1}{insert/distributions}{rdd_example_1}{\gls{rdd} map in circle geometry}%
\medfig{fig:rdd-example-2}{insert/distributions}{rdd_example_2}{\gls{rdd} map in square geometry}%

\bigskip \bigskip

The distributions in \figref{fig:rdd-example-1} and \figref{fig:rdd-example-2} can be generated and exported with the following script.

\medskip

\pyscript{insert/distributions}{rdd.py}

\newpage

\subsubsection{\glsentrytext{rrdd}}\label{sec:distribution-model-rrdd}

In a \gls{rrdd}, the dislocations are randomly positioned within sub-areas of the \gls{roi}.
A fixed number \( f \) of dislocations is sampled in each sub-area of side \( s \).
For the circular geometry, the points beyond the radius of the \gls{roi} are deleted after the random draw.
There are two ways to determine the sense of the Burgers vectors.
The first variant of the model is to set, in each sub-area, the number of Burgers vectors with a positive sense equal to the number of Burgers vectors with a negative sense.
This is the \gls{rrdde}.
The second variant of the model is to choose the sense of the Burgers vectors randomly.
This is the \gls{rrddr}.
The two variants are shown in \figref{fig:rrdd-example-1} and \figref{fig:rrdd-example-2}.
The sub-areas are indicated by the grid.

\medfig{fig:rrdd-example-1}{insert/distributions}{rrdd_example_1}{\gls{rrdde} map in circle geometry}%
\medfig{fig:rrdd-example-2}{insert/distributions}{rrdd_example_2}{\gls{rrddr} map in square geometry}%

\bigskip \bigskip

The distributions in \figref{fig:rrdd-example-1} and \figref{fig:rrdd-example-2} can be generated and exported with the following script.

\medskip

\pyscript{insert/distributions}{rrdd.py}

\newpage

\subsubsection{\glsentrytext{rcdd}}\label{sec:distribution-model-rcdd}

In a \gls{rcdd}, the dislocations are randomly positioned in cell walls.
The thickness of the walls is given by the variable \( t \) and the side of the cells by the variable \( s \).
Random positioning of points in cell walls is described in section \ref{sec:cell-random-position}.
There are three ways to determine the positions and the sense of the Burgers vectors of dislocations in this model.
In \gls{rcddr} the positions and the Burgers vectors senses are randomly distributed.
In \gls{rcdde} each cell contains a fixed number of dislocations and the Burgers vectors senses are evenly distributed in each cell.
In \gls{rcddd} the Burgers vectors with positive sense and negative sense are spaced by a distance defined with the variable \( l \).
The positions and orientations of these dipoles are then random.
\gls{rcddr} and \gls{rcddd} variants are shown in \figref{fig:rcdd-example-1} and \figref{fig:rcdd-example-2}.
The cell walls are indicated by the grid.

\medfig{fig:rcdd-example-1}{insert/distributions}{rcdd_example_1}{\gls{rcdde} map in circle geometry}%
\medfig{fig:rcdd-example-2}{insert/distributions}{rcdd_example_2}{\gls{rcddd} map in square geometry}%

\bigskip \bigskip

The distributions in \figref{fig:rcdd-example-1} and \figref{fig:rcdd-example-2} can be generated and exported with the following script.

\medskip

\pyscript{insert/distributions}{rcdd.py}

\newpage

\subsection{Boundary conditions}

To simulate the complete diffraction line profile some conditions must be applied at the boundaries of the \gls{roi}.
The latter allow to obtain realistic physical conditions for the strain and displacement fields.
Otherwise, it would be necessary to limit the analysis to the asymptotic values of the Fourier variable.

\subsubsection{\glsentrytext{isd}}

The \gls{isd} can be applied to a circular \gls{roi}.
However, the formula used here only applies to dislocations of type screw.
If there is a screw dislocation at a distance \( r \) from the center of the cylinder with radius \( R_{\gls{roi}} \) the mechanical equilibrium at the surface requires a stress field which is equivalent to the stress due to the dislocation at \( r \) plus a dislocation outside the cylinder located at \( r_{\gls{isd}} \) given in equation \eqref{eq:image-dislocation-radius}.
The image dislocation has opposite Burgers vector.

\begin{equation}\label{eq:image-dislocation-radius}
  r_{\gls{isd}} =
    \frac{R^2_{\gls{roi}}}{r}
\end{equation}

\medfig{fig:isd-example-1}{insert/boundaries}{isd_example_1}{\gls{isd} on \gls{rdd}}%
\medfig{fig:isd-example-2}{insert/boundaries}{isd_example_2}{\gls{isd} on \gls{rcddr}}%

\bigskip \bigskip

The distributions in \figref{fig:isd-example-1} and \figref{fig:isd-example-2} can be generated and exported with the following script.

\smallskip

\pyscript{insert/boundaries}{isd.py}

\newpage

\subsubsection{\glsentrytext{pbc}}

There are no closed formulas for the position of the image dislocations other than screws.
Another solution that allows to calculate the complete line profile for both edge and screw dislocations is to apply \gls{pbc} to a square geometry.
Using \gls{pbc} is more versatile than \gls{isd} but also more expensive in computing time.
With \gls{isd} the number of dislocations is doubled, but with \gls{pbc} it evolves quadratically as a function of the rank of replications around the \gls{roi}.
The number at the end of the boundary condition name specifies the rank of replications around the \gls{roi}.

\bigskip

Here the replicated dislocations are generated at the creation of the distribution.
This is useful for the representation in maps like \figref{fig:pbc-example-1} and \figref{fig:pbc-example-2} or for statistical analysis, but it is less relevant for writing the positions in a file.
To avoid transmission of large files it is possible to set a flag asking the X-ray simulation program to generate the replicated dislocations from those in the ROI.

\medfig{fig:pbc-example-1}{insert/boundaries}{pbc_example_1}{\gls{pbc}1 on \gls{rdd}}%
\medfig{fig:pbc-example-2}{insert/boundaries}{pbc_example_2}{\gls{pbc}2 on \gls{rdd}}%

\bigskip \bigskip

The distributions in \figref{fig:pbc-example-1} and \figref{fig:pbc-example-2} can be generated and exported with the following script.

\medskip

\pyscript{insert/boundaries}{pbc.py}

\newpage

\subsubsection{\glsentrytext{gbb}}

\gls{pbc} are very useful to simulate a crystal of infinite size.
However, this boundary condition poses a problem of discontinuity of some analysis functions, in particular those studying the distance between dislocations having the same Burgers vector.
This phenomenon is due to the periodicity of the position of a dislocation in the different replications of the region of interest.
If the side of the \gls{roi} is \( S_{\gls{roi}} \), then there will be systematically multiple dislocations counted at distances \( S_{\gls{roi}} \), \( \textstyle \sqrt{2} S_{\gls{roi}} \), \( 2 S_{\gls{roi}} \) etc. as shown in \figref{fig:pbc-issue}
There will also be statistically fewer dislocations before and after these distance harmonics.

\bigskip

To overcome this problem, a third boundary condition is proposed.
With the \gls{gbb}, the \gls{roi} is replicated around the boundaries but the dislocations are placed randomly.
The latter are generated following the same principles as those of the distribution model used to generate those inside the \gls{roi}.
However, the problem of large files cannot be avoided here.
As in \gls{pbc}, the number at the end of the boundary condition name specifies the rank of replications around the \gls{roi}.

\medfig{fig:pbc-issue}{insert/boundaries}{gbb_pbc_issue}{Issue with \gls{pbc}}%
\medfig{fig:gbb-example-1}{insert/boundaries}{gbb_example_1}{\gls{gbb} on \gls{rdd}}%

\bigskip \bigskip

The distribution in \figref{fig:gbb-example-1} can be generated and exported with the following script.

\medskip

\pyscript{insert/boundaries}{gbb.py}

\newpage

\subsection{Spatial analysis of distribution models}\label{sec:analysis-functions}

In sections \ref{sec:analysis-functions-K}, \ref{sec:analysis-functions-g} and \ref{sec:analysis-functions-GaGs} are presented the functions that will be used for the statistical analysis of the distribution models.
The analyzes shown in \figref{fig:KKKK-example}, \figref{fig:gggg-example} and \figref{fig:GaGs-example} can be obtained from the following script.
The execution of the latter will require a considerable amount of time on a classic computer.
For faster results it is recommended to use the module \texttt{lpa.input.parallel} to run the analysis on a supercomputer or to reduce the number of distributions in the sample.

\medskip

\pyscript{insert/analysis}{analysis.py}

\bigskip

Let \( P_2 \subset \gls{setRn} \) be a set of points in a space of dimension \( n \) and \( \vec{p_1} \in \gls{setRn} \) a point of this space.
\gls{funN} is the function that gives the number of points \( \vec{p_2} \in P_2 \) with \( \vec{p_1} \neq \vec{p_2} \) and \( \vec{p_2} \) contained in a neighborhood \( \gls{nbh}(\vec{p_1}, r) \) of radius \( r \) around \( \vec{p_1} \).
\gls{funN} is defined in equation \eqref{eq:analysis-function-N} with \gls{dstbtw} the distance between two points.

\begin{equation}\label{eq:analysis-function-N}
  \define{
    \gls{funN}
  }{
    \gls{setRn} \times \gls{pwrset} \left( \gls{setRn} \right) \times \gls{setR}_+
  }{
    \gls{setN}
  }{
    (\vec{p_1}, P_2, r)
  }{
    \sum_{\vec{p_2} \in P_2} \gls{indfun}_{\gls{dstbtw}(\vec{p_1}, \vec{p_2}) \in ]0, r]}
  }
\end{equation}

\medskip

For positions close to the edges, these neighborhoods will sometimes extend outside the \gls{roi}.
Without special boundary conditions, dislocations are not generated outside the latter.
This leads to deviations from the results that would be obtained in a crystal of infinite size.
Four approaches are proposed for the consideration of the edges.
In the first, the \gls{nec}, nothing is done.
In the second, the \gls{woa}, the ratio of the intersection of the \gls{roi} and the neighborhood is taken into account.
The third and fourth approaches (only for square geometries) are to apply the conditions \gls{pbc} or \gls{gbb} with the \gls{roi} replicated enough times for the neighborhood to be included in the grid.

\bigfig{fig:nec-and-woa}{insert/analysis}{considerations}{\gls{nec} and \gls{woa} edge considerations}

Let \( \gls{funw} \) be the weighting function for the correction at the edges of the \gls{roi} used in \gls{woa}.
The definition of \( \gls{funw} \) in equation \eqref{eq:analysis-function-w} uses the overlapping area \( \gls{area}_{\gls{nbh}(a, r) \cap \gls{roi} } \) of the \gls{roi} and the neighborhood.
The latter can be expressed with the area of intersection \circir \ defined in equation \eqref{eq:circle-circle-intersection} for circle geometries or \cirsqr \ defined in equation \eqref{eq:circle-square-intersection} for square geometries.

\begin{equation}\label{eq:analysis-function-w}
  \define{
    \gls{funw}
  }{
    \gls{setRn} \times \gls{setR}_+^*
  }{
    ]0, 1]
  }{
    (\vec{p}, r)
  }{
    \frac{\gls{area}_{\gls{nbh}(\vec{p}, r) \cap \gls{roi}}}{\gls{area}_{\gls{nbh}(\vec{p}, r)}}
  }
\end{equation}

Let \( P_+ \) be the positions of dislocations with a positive Burger vector sense and \( P_- \) the positions of dislocations with a negative Burger vector sense.
Let \( \gls{funM}_{\alpha \beta} \), with \( (\alpha, \beta) \in \{ +, - \}^2 \), be the function that gives the average number of dislocations with sense \( \beta \) around dislocations with sense \( \alpha \) in a radius \( r \).
\( \gls{funM}_{\alpha \beta} \) is defined in equations \eqref{eq:analysis-function-M-nec}, \eqref{eq:analysis-function-M-woa}, \eqref{eq:analysis-function-M-pbc} and \eqref{eq:analysis-function-M-gbb} for the different edge considerations.

\begin{minipage}{0.5\linewidth}
  \begin{align}
    \define{
      \gls{funM}_{\alpha \beta}^{\gls{nec}} &
    }{
      \gls{setR}_+
    }{
      \gls{setR}_+
    }{
      r
    }{
      \frac{\sum_{\vec{p_\alpha} \in P_\alpha} \gls{funN}(\vec{p_\alpha}, P_\beta, r)}{\sum_{\vec{p_\alpha} \in P_\alpha} 1}
    }
    \label{eq:analysis-function-M-nec}
    \\[0.3cm]
    \define{
      \gls{funM}_{\alpha \beta}^{\gls{woa}} &
    }{
      \gls{setR}_+
    }{
      \gls{setR}_+
    }{
      r
    }{
      \frac{\sum_{\vec{p_\alpha} \in P_\alpha} \gls{funN}(\vec{p_\alpha}, P_\beta, r)}{\sum_{\vec{p_\alpha} \in P_\alpha} \gls{funw} \left(\vec{p_\alpha}, r \right)}
    }
    \label{eq:analysis-function-M-woa}
  \end{align}
\end{minipage}%
\begin{minipage}{0.5\linewidth}
  \begin{align}
    \define{
      \gls{funM}_{\alpha \beta}^{\gls{pbc}} &
    }{
      \gls{setR}_+
    }{
      \gls{setR}_+
    }{
      r
    }{
      \frac{\sum_{\vec{p_\alpha} \in P_\alpha} \gls{funN}(\vec{p_\alpha}, P_\beta^{\gls{pbc}}, r)}{\sum_{\vec{p_\alpha} \in P_\alpha} 1}
    }
    \label{eq:analysis-function-M-pbc}
    \\[0.3cm]
    \define{
      \gls{funM}_{\alpha \beta}^{\gls{gbb}} &
    }{
      \gls{setR}_+
    }{
      \gls{setR}_+
    }{
      r
    }{
      \frac{\sum_{\vec{p_\alpha} \in P_\alpha} \gls{funN}(\vec{p_\alpha}, P_\beta^{\gls{gbb}}, r)}{\sum_{\vec{p_\alpha} \in P_\alpha} 1}
    }
    \label{eq:analysis-function-M-gbb}
  \end{align}
\end{minipage}%

\bigskip \bigskip

In general, to obtain the statistical behaviors, \gls{nec} and \gls{gbb} will be used in priority because \gls{pbc} produces discontinuity peaks and \gls{woa} distorts the results.
The four possibilities are nevertheless kept to illustrate the differences that can be obtained depending on the technique used and to underline the importance of these considerations.

\subsubsection{Ripley's K function}\label{sec:analysis-functions-K}

Let \( \gls{funK}_{\alpha \beta} \) with \( (\alpha, \beta) \in \{ +, - \}^2 \) be the \glsdesc{funK} \cite{IPSS2008} suitable for the statistical study of the location of dislocations depending on the sense of their Burgers vector.
\( \gls{funK}_{\alpha \beta} \) is constructed with the function \( \gls{funM}_{\alpha \beta} \) from equation \eqref{eq:analysis-function-M-nec}, \eqref{eq:analysis-function-M-woa}, \eqref{eq:analysis-function-M-pbc} or \eqref{eq:analysis-function-M-gbb} according to the desired edge consideration.
\( \gls{dst}_\beta \) is the density of dislocations with Burgers vector sense \( \beta \).
As a point of reference for the observation of the results it is recalled that for a continuous uniform distribution of undifferentiated points in an unbounded two-dimensional space \( \gls{funK}^{\gls{nec}}(r) = \pi r^2 \).

\begin{equation}\label{eq:analysis-functions-K}
  \define{
    \gls{funK}^*_{\alpha \beta}
  }{
    \gls{setR}_+
  }{
    \gls{setR}_+
  }{
    r
  }{
    \frac{1}{\gls{dst}_\beta} \gls{funM}^*_{\alpha \beta}(r)
  }
\end{equation}

\bigfig{fig:KKKK-example}{../data/stats_5e13m-2}{200000_rho5e13m-2_RDD_d5e-5nm-2_square_3200nm_S0_KKKK_GBB}{\gls{funK} on \gls{rdd}}%

\newpage

\subsubsection{Pair correlation function}\label{sec:analysis-functions-g}

Let \( \gls{fung}_{\alpha \beta} \) with \( (\alpha, \beta) \in \{ +, - \}^2 \) be the \glsdesc{fung} \cite{IPSS2008} constructed with the function \( \gls{funK}_{\alpha \beta} \) defined in equation \eqref{eq:analysis-functions-K}.
It is the derivative of the latter with respect to the area \( \gls{area}_{\gls{nbh}} \) of the neighborhood.
Differences that are difficult to perceive with \( \gls{funK}_{\alpha \beta} \) can be revealed with \( \gls{fung}_{\alpha \beta} \).
For a continuous uniform distribution of undifferentiated points in an unbounded two-dimensional space \( \gls{fung}^{\gls{nec}}(r) = 1 \).

\begin{equation}\label{eq:analysis-functions-g}
  \define{
    \gls{fung}^*_{\alpha \beta}
  }{
    \gls{setR}_+
  }{
    \gls{setR}
  }{
    r
  }{
    \frac{d \gls{funK}^*_{\alpha \beta}}{d \gls{area}_{\gls{nbh}}}(r)
  }
\end{equation}

\bigfig{fig:gggg-example}{../data/stats_5e13m-2}{200000_rho5e13m-2_RDD_d5e-5nm-2_square_3200nm_S0_gggg_GBB}{\gls{fung} on \gls{rdd}}%

\subsubsection{Symmetrical and antisymmetrical functions}\label{sec:analysis-functions-GaGs}

Gaal and Geszti \cite{GG1969} proposed two statistical functions \( G^a \) and \( G^s \).
\( G^a \) is strongly related to the expression of the total energy \gls{totnrg} per unit length. \gls{totnrg} is defined in equation \eqref{eq:total-energy-discrete} with \gls{slfnrg} the \glsdesc{slfnrg}, \( \gls{inttrm}(\gls{dstbtw}, \varphi) \) the \glsdesc{inttrm} spaced by a distance \gls{dstbtw}, \( \gls{vecb}_i \) the Burgers vector of dislocation \( i \) and \( \vec{r}_{k i} \) the relative position of dislocation \( k \) with respect to dislocation \( i \).

\begin{equation}\label{eq:total-energy-discrete}
  \gls{totnrg} =
    \gls{dst} \gls{area}_{\gls{roi}} \gls{slfnrg} + \frac{1}{2} \sum_i \sum_{k \neq i} \gls{vecb}_i \cdot \gls{vecb}_k \gls{inttrm} \left( \left| \vec{r}_{k i} \right|, \gls{ang} \left( \gls{vecb}_i, \vec{r}_{k i} \right) \right)
\end{equation}

\medskip

It was deemed appropriate to incorporate \( G^a \) and \( G^s \) into the study but some ambiguities in the definition of the functions could not be overcome.
It is proposed here to redefine these functions using \( \gls{funM}_{\alpha \beta} \)
These new functions are called \gls{funGa} and \gls{funGs} and are close to those of Gaal and Geszti on the principle.

\begin{equation}\label{eq:analysis-functions-Ga}
  \define{
    \gls{funGa}^*
  }{
    \gls{setR}_+
  }{
    \gls{setR}
  }{
    r
  }{
    \gls{dst}_+ \gls{area}_{\gls{roi}} \left( \frac{d\gls{funM}^*_{++}}{dr}(r) - \frac{d\gls{funM}^*_{+-}}{dr}(r) \right) +
    \gls{dst}_- \gls{area}_{\gls{roi}} \left( \frac{d\gls{funM}^*_{--}}{dr}(r) - \frac{d\gls{funM}^*_{-+}}{dr}(r) \right)
  }
\end{equation}

\begin{equation}\label{eq:analysis-functions-Gs}
  \define{
    \gls{funGs}^*
  }{
    \gls{setR}_+
  }{
    \gls{setR}
  }{
    r
  }{
    \gls{dst}_+ \gls{area}_{\gls{roi}} \left( \frac{d\gls{funM}^*_{++}}{dr}(r) + \frac{d\gls{funM}^*_{+-}}{dr}(r) \right) +
    \gls{dst}_- \gls{area}_{\gls{roi}} \left( \frac{d\gls{funM}^*_{--}}{dr}(r) + \frac{d\gls{funM}^*_{-+}}{dr}(r) \right)
  }
\end{equation}

\newpage

\bigfig{fig:GaGs-example}{../data/stats_5e13m-2}{200000_rho5e13m-2_RDD_d5e-5nm-2_square_3200nm_S0_GaGs_GBB}{\gls{funGa} and \gls{funGs} on \gls{rdd}}%

The new function \gls{funGa} leads to a reformulation of the total energy per unit length stored in the \gls{roi} in equation \eqref{eq:total-energy-continuous}.
The calculation can be conducted differently depending on the edge consideration * chosen.

\begin{equation}\label{eq:total-energy-continuous}
  \gls{totnrg}^* =
    \gls{dst} \gls{area}_{\gls{roi}} \gls{slfnrg}^* + \frac{1}{2} |\gls{vecb}|^2 \int_{r} \gls{funGa}^* (r) \gls{expval} \left( \gls{inttrm}^* (r, \phi) \right) dr
\end{equation}

\subsection{Closed-form expressions}

In this section are proposed closed-form expressions for the analysis functions and the stored energy per unit length \gls{totnrg}.
The expressions of the self and interaction energy per unit length \gls{slfnrg} and \gls{intnrg} of dislocations can be found in the literature \cite{ES1951} \cite{N1967} \cite{HB2011}.
Let \gls{shrmod} be the \glsdesc{shrmod}, \( r_0 \) the core radius, \gls{pssrat} the \glsdesc{pssrat} and \( \varphi \) is the angle between the Burgers vector \gls{vecb} of a dislocation and the position of the second dislocation of the interacting pair.
The force exerted by a dislocation 1 on a dislocation 2 is given in equation \eqref{eq:force}.

\medskip

\begin{equation}\label{eq:force}
  \vec{F}_{12} =
    \begin{cases}
      \gls{vecb}_1 \cdot \gls{vecb}_2 \frac{\gls{shrmod}}{2 \pi r} \ur
      & \text{for screw dislocations}
      \\[0.3cm]
      \gls{vecb}_1 \cdot \gls{vecb}_2 \frac{\gls{shrmod}}{2 \pi r (1 - \gls{pssrat})} \left( \ur + \sin \left( 2 \varphi \right) \uphi \right)
      & \text{for edge dislocations}
    \end{cases}
\end{equation}

\bigfig{fig:forces}{insert/expressions}{forces}{Force exerted between dislocations of type screw and edge}%

\newpage

Let \( \mathcal{B}_{\gls{roi}} \) be the the function that gives the expected value of the intersection of the \gls{roi} with a neighbohood \( \gls{nbh}(x, y, r) \) of radius \( r \) around a point \( ( x, y ) \) randomly chosen in the \gls{roi}.
Let \( S_{\gls{roi}} \) be the side of the \gls{roi} for square geometries and \( R_{\gls{roi}} \) be its radius for circular geometries.
The definition of \( \mathcal{B}_{\gls{roi}} \) in equation \eqref{eq:function-B} uses \( \gls{expval} \left( \cirsqr \right) \) and \( \gls{expval} \left( \circir \right) \) defined in equation \eqref{eq:mean-circle-circle-intersection} and \eqref{eq:mean-circle-square-intersection}.

\begin{equation}\label{eq:function-B}
  \mathcal{B}_{\gls{roi}}(r) =
    \gls{expval} \left( \gls{area}_{\gls{roi} \cap \gls{nbh}(X, Y, r)} \right) =
    \begin{cases}
      \gls{expval} \left( \cirsqr(r, S_{\gls{roi}}, X, Y) \right)
      & \text{for square geometry} \\
      \gls{expval} \left( \circir(r, R_{\gls{roi}}, D) \right)
      & \text{for circular geometry} \\
    \end{cases}
\end{equation}

\bigskip

The self-energy per unit length \gls{slfnrg} of a dislocation, stored outside the core, is given by integration over the region of interest.
The strain energy per unit volume is one-half the product of stress times strain for each component.
For the edge consideration \gls{woa} it is used a cylindrical shell of radius \( r \) and surface \( 2 \pi r dr \) in equation \eqref{eq:self-energy-woa}.
With \gls{nec} in equation \eqref{eq:self-energy-nec}, the area of the cylindrical shell is expressed with \( \mathcal{B}_{\gls{roi}} \) to correspond to the area actually included in the \gls{roi}.

\begin{equation}
  \gls{slfnrg} =
    \int \frac{1}{2} \sum_i \sum_k \sigma_i e_k d\gls{area}
\end{equation}

\begin{align}
  \gls{slfnrg}^{\gls{woa}} &=
    \int_{r=r_0}^{R_{\gls{roi}}} \frac{2 \gls{shrmod}}{1 - \gls{indfun}_{\mathrm{edge}} \gls{pssrat}} \left( \frac{| \gls{vecb} |}{4 \pi r} \right)^2 2 \pi r dr
    \label{eq:self-energy-woa}
  \\
  \gls{slfnrg}^{\gls{nec}} &=
    \int_{r=r_0}^{+\infty} \frac{2 \gls{shrmod}}{1 - \gls{indfun}_{\mathrm{edge}} \gls{pssrat}} \left( \frac{| \gls{vecb} |}{4 \pi r} \right)^2 \frac{d \mathcal{B}_{\gls{roi}}}{dr} (r) dr
    \label{eq:self-energy-nec}
\end{align}

\bigskip

The interaction enegy per unit length \gls{intnrg} between two dislocations is defined in equation \eqref{eq:interaction-energy} as the work done by displacing the dislocation 2 from the border of the \gls{roi} to its current location is the presence of the stress field of the dislocation 1.

\begin{equation}\label{eq:interaction-energy}
  \gls{intnrg} =
  \left( \gls{vecb}_1 \cdot \gls{vecb}_2 \right) \gls{inttrm}(r, \varphi) =
  - \int \vec{F}_{1 2} \cdot \ux dx
\end{equation}

\begin{align}
  \gls{inttrm}^{\gls{woa}} &=
    - \frac{\gls{shrmod}}{2 \pi \left(1 - \gls{indfun}_{\mathrm{edge}} \gls{pssrat} \right)} \left( \int \frac{1}{r} dr - \gls{indfun}_{\mathrm{edge}} r^2 \cos \left( 2 \varphi \right) \int \frac{1}{r^3} dr \right)
    \label{eq:interaction-term-woa}
  \\
  \gls{inttrm}^{\gls{nec}} &=
    - \frac{\gls{shrmod}}{2 \pi \left(1 - \gls{indfun}_{\mathrm{edge}} \gls{pssrat} \right)} \left( \int \frac{1}{r} \frac{1}{2 \pi r} \frac{d \mathcal{B}_{\gls{roi}}}{dr} (r) dr - \gls{indfun}_{\mathrm{edge}} r^2 \cos \left( 2 \varphi \right) \int \frac{1}{r^3} \frac{1}{2 \pi r} \frac{d \mathcal{B}_{\gls{roi}}}{dr} (r) dr \right)
    \label{eq:interaction-term-nec}
\end{align}

\bigskip

In \gls{rdd} and \gls{rrdd} there is no reason why there should be a favored direction in the relative location of dislocations.
This leads to the final formulation of the interaction term in equations \eqref{eq:average-interaction-term-woa} and \eqref{eq:average-interaction-term-nec}.

\begin{align}
  \gls{expval} \left( \gls{inttrm}^{\gls{woa}}(r, \phi) \right) &=
    \frac{\gls{shrmod}}{2 \pi \left(1 - \gls{indfun}_{\mathrm{edge}} \gls{pssrat} \right)} \ln \left( \frac{R_{\gls{roi}}}{r} \right)
    \label{eq:average-interaction-term-woa}
  \\
  \gls{expval} \left( \gls{inttrm}^{\gls{nec}}(r, \phi) \right) &=
    \frac{\gls{shrmod}}{2 \pi \left(1 - \gls{indfun}_{\mathrm{edge}} \gls{pssrat} \right)} \int_{x=r}^{+ \infty} \frac{1}{2 \pi x^2} \frac{d \mathcal{B}_{\gls{roi}}}{dx} (x) dx
    \label{eq:average-interaction-term-nec}
\end{align}

\medskip

Finally, the total energy per unit length stored in the \gls{roi} is given in equations \eqref{eq:total-energy-woa} and \eqref{eq:total-energy-nec} for edge consideration \gls{woa} and \gls{nec}.

\begin{align}
  \gls{totnrg}^{\gls{woa}} &=
    \frac{\gls{shrmod} | \gls{vecb} |^2}{4 \pi \left(1 - \gls{indfun}_{\mathrm{edge}} \gls{pssrat} \right)} \left( \gls{dst} \gls{area}_{\gls{roi}} \ln \left( \frac{R_{\gls{roi}}}{r_0} \right) + \int_{r=r_0}^{R_{\gls{roi}}} \gls{funGa}^{\gls{woa}} (r) \ln \left( \frac{R_{\gls{roi}}}{r} \right) dr \right)
    \label{eq:total-energy-woa}
  \\
  \gls{totnrg}^{\gls{nec}} &=
    \frac{\gls{shrmod} | \gls{vecb} |^2}{4 \pi \left(1 - \gls{indfun}_{\mathrm{edge}} \gls{pssrat} \right)} \left( \gls{dst} \gls{area}_{\gls{roi}} \int_{r=r_0}^{+\infty} \frac{1}{2 \pi r^2} d \mathcal{B}_{\gls{roi}}(r) + \int_{r=r_0}^{+\infty} \gls{funGa}^{\gls{nec}} (r) \int_{x=r}^{+\infty} \frac{1}{2 \pi x^2} d \mathcal{B}_{\gls{roi}}(x) dr \right)
    \label{eq:total-energy-nec}
\end{align}

\subsubsection{\glsentrytext{rdd}}

The equations \eqref{eq:M-RDD-NEC}, \eqref{eq:M-RDD-WOA} and \eqref{eq:M-RDD-GBB} give the expression of \( \gls{funM}_{\alpha \beta} \) in \gls{rdd} for the edge considerations \gls{nec}, \gls{woa} and \gls{gbb}.
From these expressions it is possible to establish the expressions of the other analysis functions according to their definitions in section \ref{sec:analysis-functions}.
In general, \( \gls{funM}_{\alpha \alpha} \) is not equal to \( \gls{funM}_{\alpha \beta} \) because the dislocation from which we look must be subtracted from the dislocations potentially observed.

\begin{align}
  \gls{funM}_{\alpha \beta}^{\gls{nec}} (r) &=
    \left( \frac{\gls{dst}}{2} - \frac{\gls{indfun}_{\alpha = \beta}}{\gls{area}_{\gls{roi}}} \right) \mathcal{B}_{\gls{roi}}(r)
    \label{eq:M-RDD-NEC}
  \\[3mm]
  \gls{funM}_{\alpha \beta}^{\gls{woa}} (r) &=
    \gls{funM}_{\alpha \beta}^{\gls{nec}} (r) \frac{\gls{area}_{\gls{nbh}(r)}}{\mathcal{B}_{\gls{roi}}(r)}
    \label{eq:M-RDD-WOA}
  \\[3mm]
  \gls{funM}_{\alpha \beta}^{\gls{gbb}} (r) &=
    \gls{funM}_{\alpha \beta}^{\gls{nec}} (r) + \frac{\gls{dst}}{2} \left( \gls{area}_{\gls{nbh}(r)} - \mathcal{B}_{\gls{roi}}(r) \right)
    \label{eq:M-RDD-GBB}
\end{align}

\bigskip

\figref{fig:RDD-g-NEC}, \figref{fig:RDD-g-WOA} and \figref{fig:RDD-g-GBB} show the function \( \gls{fung}_{\alpha \beta} \) for the different edge consideration and compare the simulation results to the closed-form expressions obtained from equations \eqref{eq:M-RDD-NEC}, \eqref{eq:M-RDD-WOA} and \eqref{eq:M-RDD-GBB}.

\smallfig{fig:RDD-g-NEC}{insert/expressions/plots}{square_RDD_g_NEC}{\( \gls{fung}^{\gls{nec}} \) on \gls{rdd}}%
\smallfig{fig:RDD-g-WOA}{insert/expressions/plots}{square_RDD_g_WOA}{\( \gls{fung}^{\gls{woa}} \) on \gls{rdd}}%
\smallfig{fig:RDD-g-GBB}{insert/expressions/plots}{square_RDD_g_GBB}{\( \gls{fung}^{\gls{gbb}} \) on \gls{rdd}}%

\bigskip \bigskip

Equations \eqref{eq:Ga-RDD-NEC-GBB} and \eqref{eq:Ga-RDD-WOA} give the expression of \gls{funGa} and \figref{fig:RDD-Ga-NEC}, \figref{fig:RDD-Ga-WOA} and \figref{fig:RDD-Ga-GBB} compare the simulation results to the closed-form expressions.

\begin{align}
  \gls{funGa}^{\gls{nec}} (r) =
  \gls{funGa}^{\gls{gbb}} (r) &=
    - \gls{dst} \frac{d \mathcal{B}_{\gls{roi}}}{dr}(r)
    \label{eq:Ga-RDD-NEC-GBB}
  \\[3mm]
  \gls{funGa}^{\gls{woa}} (r) &=
    - \gls{dst} 2 \pi r
    \label{eq:Ga-RDD-WOA}
\end{align}

\smallfig{fig:RDD-Ga-NEC}{insert/expressions/plots}{square_RDD_Ga_NEC}{\( \gls{funGa}^{\gls{nec}} \) on \gls{rdd}}%
\smallfig{fig:RDD-Ga-WOA}{insert/expressions/plots}{square_RDD_Ga_WOA}{\( \gls{funGa}^{\gls{woa}} \) on \gls{rdd}}%
\smallfig{fig:RDD-Ga-GBB}{insert/expressions/plots}{square_RDD_Ga_GBB}{\( \gls{funGa}^{\gls{gbb}} \) on \gls{rdd}}%

\bigskip \bigskip

With equation \eqref{eq:Ga-RDD-WOA} and \eqref{eq:total-energy-woa} it is possible to give a simple expression of the energy for the edge consideration \gls{woa}.
The effective cut-off radius is given in equation \eqref{eq:cut-rad-rdd}.
The latter is proportional to \( R_{\gls{roi}} \).
As shown in \figref{fig:circle-RDD-energy-WOA}, \figref{fig:circle-RDD-energy-NEC} and \figref{fig:square-RDD-energy-NEC} the total energy per unit volume is divergent.

\begin{equation}
  \gls{totnrg}^{\gls{woa}} =
    \frac{\gls{shrmod} | \gls{vecb} |^2}{4 \pi \gamma} \left( \ln \left( \frac{R_{\gls{roi}}}{r_0 \sqrt{e}} \right) + \left( \frac{r_0}{R_{\gls{roi}}} \right)^2 \ln \left( \frac{R_{\gls{roi}} \sqrt{e}}{r_0} \right) \right)
    \ \underset{\frac{r_0}{R_{\gls{roi}}} \rightarrow 0}{\sim} \
    \frac{\gls{shrmod} | \gls{vecb} |^2}{4 \pi \gamma} \ln \left( \frac{R_{\gls{roi}}}{r_0 \sqrt{e}} \right)
\end{equation}

\begin{equation}\label{eq:cut-rad-rdd}
\gls{cutrad}^{\gls{woa}} = \frac{R_{\gls{roi}}}{\sqrt{e}}
\end{equation}

\smallfig{fig:circle-RDD-energy-WOA}{insert/expressions/plots}{circle_RDD_energy_WOA}{\gls{woa} circle \gls{rdd}}%
\smallfig{fig:circle-RDD-energy-NEC}{insert/expressions/plots}{circle_RDD_energy_NEC}{\gls{nec} circle \gls{rdd}}%
\smallfig{fig:square-RDD-energy-NEC}{insert/expressions/plots}{square_RDD_energy_NEC}{\gls{nec} square \gls{rdd}}%

\subsubsection{\glsentrytext{rrdd}}

Here is presented the method for finding the closed-form expression of the energy stored in the distributions of type \gls{rrdde} and \gls{rrddr}.
Let \( s \) be the side of the subareas and \( \mathcal{Z} \) the set of subareas in the \gls{roi}.
To calculate the number of dislocations with sense \( \beta \) around dislocations with sense \( \alpha \) within a radius \( r \), one must start by placing oneself on one of the dislocations with sense \( \alpha \).
Let \( (x, y) \) be the position of the selected dislocation relative to the lower left corner of the subarea \( \zeta \in \mathcal{Z} \) in which it is located.
Let \( \gls{nbh}(x, y, r) \) be a neighbohood of radius \( r \) centered in \( (x, y) \).

\bigfig{fig:overlapping:rrdd}{insert/overlapping}{rrdd}{Selected dislocation in the subarea \( \zeta \) and its neighborhood.}

Let \( \tau_{\alpha \beta} \) be the number of dislocations with Burgers vector sense \( \beta \) that can be observed in a subarea \( Z \) by a dislocation with Burgers vector sense \( \alpha \) outside \( Z \).
Let \( \eta_{\alpha \beta} \) be the number of dislocations with sense \( \beta \) that can be observed in a subarea \( Z \) by a dislocation with sense \( \alpha \) inside \( Z \).
Let \( f \) be the number of dislocations in a subarea.
The position of the chosen dislocation of sense \( \alpha \) is random in a subarea.
\( X \) and \( Y \) follow a continuous uniform distribution law \( U(0, s) \).

\newpage

The function \( \gls{funM}_{\alpha \beta} \) can be expressed as a sum of the contributions of the intersection of each subarea \( Z \) with the neighborhood \( \gls{nbh}(x, y, r) \) in equation \eqref{eq:RRDD-M}.
It is the expression \eqref{eq:RRDD-M-easy} that will be easiest to use.

\begin{align}
  \gls{funM}_{\alpha \beta}^{\gls{nec}} (r) &=
    \gls{expval} \left( \eta_{\alpha \beta} \frac{\gls{area}_{\zeta \cap \gls{nbh}(X, Y, r)}}{\gls{area}_\zeta} + \sum_{Z \in \mathcal{Z} \neq \zeta} \tau_{\alpha \beta} \frac{\gls{area}_{Z \cap \gls{nbh}(X, Y, r)}}{\gls{area}_Z} \right)
    \label{eq:RRDD-M}
  \\[3mm]
  &=
    \gls{expval} \left( \left( \eta_{\alpha \beta} - \tau_{\alpha \beta} \right) \frac{\gls{area}_{\zeta \cap \gls{nbh}(X, Y, r)}}{\gls{area}_\zeta} + \tau_{\alpha \beta} \sum_{Z \in \mathcal{Z}} \frac{\gls{area}_{Z \cap \gls{nbh}(X, Y, r)}}{\gls{area}_Z} \right)
  \\[3mm]
  &=
    \frac{1}{s^2} \left( \left( \eta_{\alpha \beta} - \tau_{\alpha \beta} \right) \gls{expval} \left( \cirsqr(r, s, X, Y) \right) + \tau_{\alpha \beta} \gls{expval} \left( \sum_{Z \in \mathcal{Z}} \gls{area}_{Z \cap \gls{nbh}(X, Y, r)} \right) \right)
    \label{eq:RRDD-M-easy}
\end{align}

\medskip

From equation \eqref{eq:RRDD-M-easy} it is possible to define \( \gls{funM}_{\alpha \beta} \) for the edge considerations \gls{nec}, \gls{woa} and \gls{gbb} in equations \eqref{eq:RRDD-M-NEC}, \eqref{eq:RRDD-M-WOA} and \eqref{eq:RRDD-M-GBB}.

\begin{align}
  \gls{funM}^{\gls{nec}}_{\alpha \beta} (r) &=
    \frac{\left( \eta_{\alpha \beta} - \tau_{\alpha \beta} \right) \gls{expval} \left( \cirsqr(r, s, X, Y) \right) + \tau_{\alpha \beta} \mathcal{B}_{\gls{roi}}(r)}{s^2}
    \label{eq:RRDD-M-NEC}
  \\[0.1cm]
  \gls{funM}^{\gls{woa}}_{\alpha \beta} (r) &=
    \gls{funM}^{\gls{nec}}_{\alpha \beta} (r) \frac{\gls{area}_{\gls{nbh}(r)}}{\mathcal{B}_{\gls{roi}}(r)}
    \label{eq:RRDD-M-WOA}
  \\[0.1cm]
  \gls{funM}^{\gls{gbb}}_{\alpha \beta} (r) &=
    \gls{funM}^{\gls{nec}}_{\alpha \beta} (r) + \frac{\gls{dst}}{2} \left( \gls{area}_{\gls{nbh}(r)} - \mathcal{B}_{\gls{roi}}(r) \right)
    \label{eq:RRDD-M-GBB}
\end{align}

\subsubsection{\glsentrytext{rrdde}}

The coefficients \( \tau_{\alpha \beta} \) and \( \eta_{\alpha \beta} \) for the distribution model \gls{rrdde} are given in equation \eqref{eq:tau-rrdde} and \eqref{eq:eta-rrdde}.
The latter are not equal because the dislocation from which we look must be subtracted from the dislocations potentially observed within the subarea.

\begin{align}
  \tau_{\alpha \beta} &=
    \frac{f}{2}
    \label{eq:tau-rrdde}
  \\[3mm]
  \eta_{\alpha \beta} &=
    \frac{f}{2} - \gls{indfun}_{\alpha = \beta}
    \label{eq:eta-rrdde}
\end{align}

\bigskip

From equations \eqref{eq:RRDD-M-NEC}, \eqref{eq:RRDD-M-WOA}, \eqref{eq:RRDD-M-GBB} \eqref{eq:tau-rrdde}, \eqref{eq:eta-rrdde} and \eqref{eq:analysis-functions-g} it is possible to express the function \gls{fung} for each edge consideration.
The closed form of the latter is compared to simulation results in \figref{fig:RRDD-E-g-NEC}, \figref{fig:RRDD-E-g-WOA} and \figref{fig:RRDD-E-g-GBB}.

\smallfig{fig:RRDD-E-g-NEC}{insert/expressions/plots}{square_RRDD-E_g_NEC}{\( \gls{fung}^{\gls{nec}} \) on \gls{rrdde}}%
\smallfig{fig:RRDD-E-g-WOA}{insert/expressions/plots}{square_RRDD-E_g_WOA}{\( \gls{fung}^{\gls{woa}} \) on \gls{rrdde}}%
\smallfig{fig:RRDD-E-g-GBB}{insert/expressions/plots}{square_RRDD-E_g_GBB}{\( \gls{fung}^{\gls{gbb}} \) on \gls{rrdde}}%

\newpage

\figref{fig:RRDD-E-Ga-NEC}, \figref{fig:RRDD-E-Ga-WOA} and \figref{fig:RRDD-E-Ga-GBB} show the plot of \gls{funGa} expressed for each edge consideration in equations \eqref{eq:Ga-RRDDE-NEC-GBB} and \eqref{eq:Ga-RRDDE-WOA}.

\begin{align}
  \gls{funGa}^{\gls{nec}} (r) =
  \gls{funGa}^{\gls{gbb}} (r) &=
    - \frac{\gls{dst} \gls{area}_{\gls{roi}}}{s^2} \frac{d \gls{expval} \left( \cirsqr(r, s, X, Y) \right)}{dr}
    \label{eq:Ga-RRDDE-NEC-GBB}
  \\[3mm]
  \gls{funGa}^{\gls{woa}} (r) &=
    - \frac{\gls{dst} \gls{area}_{\gls{roi}}}{s^2} \frac{d}{dr} \left( \frac{\gls{expval} \left( \cirsqr(r, s, X, Y) \right) \pi r^2}{ \mathcal{B}_{\gls{roi}}(r)} \right)
    \label{eq:Ga-RRDDE-WOA}
\end{align}

\smallfig{fig:RRDD-E-Ga-NEC}{insert/expressions/plots}{square_RRDD-E_Ga_NEC}{\( \gls{funGa}^{\gls{nec}} \) on \gls{rrdde}}%
\smallfig{fig:RRDD-E-Ga-WOA}{insert/expressions/plots}{square_RRDD-E_Ga_WOA}{\( \gls{funGa}^{\gls{woa}} \) on \gls{rrdde}}%
\smallfig{fig:RRDD-E-Ga-GBB}{insert/expressions/plots}{square_RRDD-E_Ga_GBB}{\( \gls{funGa}^{\gls{gbb}} \) on \gls{rrdde}}%

\bigskip

\figref{fig:circle-RRDD-E-energy-WOA} to \figref{fig:energy-3D-NEC} show the behavior of the total energy stored per unit volume. It seems to depend little on \gls{cutrad} but more on \( s \).

\smallfig{fig:circle-RRDD-E-energy-WOA}{insert/expressions/plots}{circle_RRDD-E_energy_WOA}{\gls{woa} circle \gls{rrdde}}%
\smallfig{fig:circle-RRDD-E-energy-NEC}{insert/expressions/plots}{circle_RRDD-E_energy_NEC}{\gls{nec} circle \gls{rrdde}}%
\smallfig{fig:square-RRDD-E-energy-NEC}{insert/expressions/plots}{square_RRDD-E_energy_NEC}{\gls{nec} square \gls{rrdde}}%

\medfig{fig:energy-3D-WOA}{insert/expressions/plots}{circle_energy3D_WOA}{\gls{woa} circle \gls{rrdde}}%
\medfig{fig:energy-3D-NEC}{insert/expressions/plots}{circle_energy3D_WOA}{\gls{nec} circle \gls{rrdde}}%

\subsubsection{\glsentrytext{rrddr}}

The coefficients \( \tau_{\alpha \beta} \) and \( \eta_{\alpha \beta} \) for the distribution model \gls{rrddr} can be obtained by enumeration.
In equations \eqref{eq:tau-rrddr} and \eqref{eq:eta-rrddr}, \( i \) is the number of dislocation with Burgers vector sense \( \beta \) in the observed subarea.

\begin{align}
  \tau_{\alpha \beta} &=
    \begin{cases}
      \sum_{i=1}^f \binom{f}{i} \frac{i}{f 2^{f-1}} \left( \frac{\gls{dst}}{2} \gls{area}_{\gls{roi}} - i \right) \frac{s^2}{\gls{area}_{\gls{roi}} - s^2} & \text{for } \alpha = \beta \\[4mm]
      \sum_{i=1}^f \binom{f}{i} \frac{i}{f 2^{f-1}} \left( \frac{\gls{dst}}{2} \gls{area}_{\gls{roi}} - (f - i) \right) \frac{s^2}{\gls{area}_{\gls{roi}} - s^2} & \text{for } \alpha \neq \beta
    \end{cases}
    \label{eq:tau-rrddr}
  \\[3mm]
  \eta_{\alpha \beta} &=
    \begin{cases}
      \sum_{i=1}^f \binom{f}{i} \frac{i}{f 2^{f-1}} (i - 1)
      & \text{for } \alpha = \beta \\[4mm]
      \sum_{i=1}^f \binom{f}{i} \frac{i}{f 2^{f-1}} (f - 1)
      & \text{for } \alpha \neq \beta
    \end{cases}
    \label{eq:eta-rrddr}
\end{align}

\medskip

The expression of \( \tau_{\alpha \beta} \) and \( \eta_{\alpha \beta} \) can be simplified in equations \eqref{eq:tau-rrddr-simplified} and \eqref{eq:eta-rrddr-simplified}.

\begin{align}
  \tau_{\alpha \beta} &=
    \frac{1}{2} \frac{s^2}{\gls{area}_{\gls{roi}}-s^2} \left( \gls{dst} \gls{area}_{\gls{roi}} - f + (-1)^{\gls{indfun}_{\alpha = \beta}} \right)
    \label{eq:tau-rrddr-simplified}
  \\[3mm]
  \eta_{\alpha \beta} &=
    \frac{f-1}{2}
    \label{eq:eta-rrddr-simplified}
\end{align}

\medskip

From equations \eqref{eq:RRDD-M-NEC}, \eqref{eq:RRDD-M-WOA}, \eqref{eq:RRDD-M-GBB} \eqref{eq:tau-rrddr-simplified}, \eqref{eq:eta-rrddr-simplified} and \eqref{eq:analysis-functions-g} it is possible to express the function \gls{fung} for each edge consideration.
The closed form of the latter is compared to simulation results in \figref{fig:RRDD-R-g-NEC}, \figref{fig:RRDD-R-g-WOA} and \figref{fig:RRDD-R-g-GBB}.

\smallfig{fig:RRDD-R-g-NEC}{insert/expressions/plots}{square_RRDD-R_g_NEC}{\( \gls{fung}^{\gls{nec}} \) on \gls{rrddr}}%
\smallfig{fig:RRDD-R-g-WOA}{insert/expressions/plots}{square_RRDD-R_g_WOA}{\( \gls{fung}^{\gls{woa}} \) on \gls{rrddr}}%
\smallfig{fig:RRDD-R-g-GBB}{insert/expressions/plots}{square_RRDD-R_g_GBB}{\( \gls{fung}^{\gls{gbb}} \) on \gls{rrddr}}%

\bigskip

\figref{fig:RRDD-R-Ga-NEC}, \figref{fig:RRDD-R-Ga-WOA} and \figref{fig:RRDD-R-Ga-GBB} show the plot of \gls{funGa} expressed for each edge consideration in equations \eqref{eq:Ga-RRDDR-NEC-GBB} and \eqref{eq:Ga-RRDDR-NEC-GBB}.

\begin{align}
  \gls{funGa}^{\gls{nec}} (r) =
  \gls{funGa}^{\gls{gbb}} (r) &=
    \frac{\gls{dst} \gls{area}_{\gls{roi}}}{ \gls{area}_{\gls{roi}} - s^2} \frac{d}{dr} \left( \gls{expval} \left( \cirsqr(r, s, X, Y) \right) - \mathcal{B}_{\gls{roi}} (r) \right)
    \label{eq:Ga-RRDDR-NEC-GBB}
  \\[3mm]
  \gls{funGa}^{\gls{woa}} (r) &=
    \frac{\gls{dst} \gls{area}_{\gls{roi}}}{ \gls{area}_{\gls{roi}} - s^2} \frac{d}{dr} \left( \frac{\gls{expval} \left( \cirsqr(r, s, X, Y) \right)}{\mathcal{B}_{\gls{roi}}(r)} - \pi r^2 \right)
    \label{eq:Ga-RRDDR-WOA}
\end{align}

\smallfig{fig:RRDD-R-Ga-NEC}{insert/expressions/plots}{square_RRDD-R_Ga_NEC}{\( \gls{funGa}^{\gls{nec}} \) on \gls{rrddr}}%
\smallfig{fig:RRDD-R-Ga-WOA}{insert/expressions/plots}{square_RRDD-R_Ga_WOA}{\( \gls{funGa}^{\gls{woa}} \) on \gls{rrddr}}%
\smallfig{fig:RRDD-R-Ga-GBB}{insert/expressions/plots}{square_RRDD-R_Ga_GBB}{\( \gls{funGa}^{\gls{gbb}} \) on \gls{rrddr}}%

\bigskip

\figref{fig:circle-RRDD-R-energy-WOA}, \figref{fig:circle-RRDD-R-energy-NEC} and \figref{fig:square-RRDD-R-energy-NEC} show the behavior of the total energy stored per unit volume. It does not diverge.

\smallfig{fig:circle-RRDD-R-energy-WOA}{insert/expressions/plots}{circle_RRDD-R_energy_WOA}{\gls{woa} circle \gls{rrddr}}%
\smallfig{fig:circle-RRDD-R-energy-NEC}{insert/expressions/plots}{circle_RRDD-R_energy_NEC}{\gls{nec} circle \gls{rrddr}}%
\smallfig{fig:square-RRDD-R-energy-NEC}{insert/expressions/plots}{square_RRDD-R_energy_NEC}{\gls{nec} square \gls{rrddr}}%

\newpage

\subsection{Program}

The first program concerns everything that happens before the X-ray diffraction simulation.
It has notably allowed the generation of the previous figures with the corresponding scripts.
What is called \textit{input data} is what is transmitted to the X-ray diffraction simulation program (it is not the input of the current program).

\subsubsection{Repository}

Useful information can be found on the project page: \github{lpa-input}

\subsubsection{Installation}

The program can be installed or updated with the following command:

\pipinstall{lpa-input}

\subsubsection{Input data}

The structure of the input data is shown in the sample file below.
In the header are indications for the simulation program and below the positions and Burgers vector of the generated dislocations.
A file of this kind can be written for each randomly generated distribution.
The files resulting from the same set of generation parameter (\gls{roi} shape, distribution model, variant, density...) are called \textit{sample} and can be grouped in a directory.
Then, in the following programs, the analysis will be performed on the files of this sample and the results will be averaged to obtain representative data associated with the starting parameters.

\txtlst{insert/input}{input\_data\_structure.txt}
